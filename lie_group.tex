\documentclass[12pt]{article}

\usepackage{xr-hyper}

\usepackage{geometry}
\usepackage{hyperref}
 
\hypersetup{
  colorlinks = true,
  linkcolor = blue,
  citecolor = red,
  urlcolor = teal
}

\externaldocument[calculus-]{calculus}
\externaldocument[group-]{group}
\externaldocument[field-]{field}
\externaldocument[ode-]{ode}
\externaldocument[functional-analysis-]{functional_analysis}
\externaldocument[pde-]{pde}
\externaldocument[topology-]{topology}
\externaldocument[smooth-manifold-]{smooth_manifold}
\externaldocument[riemannian-manifold-]{riemannian_manifold}
\externaldocument[lie-group-]{lie_group}

\usepackage[leqno]{amsmath}
\usepackage{amssymb}
\usepackage[centercolon]{mathtools}
\usepackage{stmaryrd}
\usepackage{wasysym}
\usepackage{amsthm}
\usepackage{mathrsfs}
\usepackage{bm}

\usepackage{graphicx}
\usepackage{float}

\usepackage{tikz}
\usepackage{tikz-cd}

\geometry{
  paper=a4paper,
  top=3cm,
  inner=2.54cm,
  outer=2.54cm,
  bottom=3cm,
  headheight=6ex,
  headsep=6ex,
  twoside,
  asymmetric
}{\relax}

\usepackage{fancyhdr}
\pagestyle{fancy}
\renewcommand{\sectionmark}[1]{\markright{#1}}
\fancyhf{}
\fancyhead[EC]{\footnotesize{\leftmark}\vspace{1mm}} %页眉部分偶数页显示章
\fancyhead[OC]{\footnotesize{\rightmark}\vspace{1mm}} %页眉部分奇数页显示节
\fancyhead[LE,RO]{{\footnotesize \thepage}\vspace{1mm}} %奇数页右边, 偶数页左边显示页码
\fancyhead[RE,LO]{}
% \fancyfoot[C]{\NTdraftstring}
\renewcommand{\headrulewidth}{0pt} %删去页眉横线
\renewcommand{\footrulewidth}{0pt} %删去页脚横线
\addtolength{\headheight}{0.5pt}

\numberwithin{equation}{subsection}

\theoremstyle{plain}
\newtheorem{theorem}{Theorem}[section]
\newtheorem{lemma}[theorem]{Lemma}
\newtheorem{proposition}[theorem]{Proposition}
\newtheorem{corollary}[theorem]{Corollary}

\theoremstyle{definition}
\newtheorem{definition}[theorem]{Definition}
\newtheorem{example}[theorem]{Example}

\theoremstyle{remark}
\newtheorem{remark}[theorem]{Remark}

\newcommand{\dif}{\mathop{}\!\mathrm{d}}
\renewcommand{\hom}{\operatorname{Hom}}


\begin{document}
\title{Lie Group}
\label{chapter-Lie-group}

\section{The Exponential Map}
\label{section-the-exponential-map}

\subsection{One-Parameter Subgroup and the Exponential Map}
\label{subsection-one-parameter-subgroup-and-the-exponential-map}

\paragraph{One-Parameter Subgroups}

\begin{definition}
  \label{definition-one-parameter-subgroup}
  A \emph{one-parameter subgroup of} \( G \) is defined to be a Lie homomorphism \( g: \mathbb{R} \to G \) with \( \mathbb{R} \) considered as a Lie group under addition.
\end{definition}

\begin{theorem}
  \label{theorem-characterization-one-parameter-subgroup}
  Let \( G \) be a Lie group.
  The one-parameter subgroups of \( G \) are precisely the maximal integral curves of left-invariant vector field starting at the identity.
\end{theorem}

\begin{definition}
  \label{definition-generated-one-parameter-subgroup}
  Given \( X \in \operatorname{Lie}(G) \), the one-parameter subgroup determined by \( X \) in this way is called the \emph{one-parameter subgroup generated by} \( X \).
\end{definition}

The one-parameter subgroups of \( \operatorname{GL}(n, \mathbb{R}) \) are not hard to compute explicitly.

\begin{proposition}
  \label{proposition-one-parameter-subgroup-of-GL}
  For any \( A \in \mathfrak{gl}(n, \mathbb{R}) \), let
  \[
    e^A = \sum_{k = 0}^{\infty} \frac{1}{k!} A^k = I_n + A + \frac{1}{2} A^2 + \cdots.
  \]
  This series converges to an invertible matrix \( e^A \in \operatorname{GL}(n, \mathbb{R}) \), and the one-parameter subgroup of \( \operatorname{GL}(n, \mathbb{R}) \) generated by \( A \in \mathfrak{gl}(n, \mathbb{R}) \) is \( \gamma(t) = e^{tA} \).
\end{proposition}

We would like to compute the one-parameter subgroups of \( \operatorname{GL}(n, \mathbb{R}) \), such as \( O(n) \).

\begin{proposition}
  \label{proposition-one-parameter-subgroup-of-subgroups-of-GL}
  Suppose \( G \) is a Lie group and \( H \subseteq G \) is a Lie subgroup.
  The one-parameter subgroups of \( H \) are precisely those one-parameter subgroups of \( G \) whose initial velocities lie in \( T_e H \).
\end{proposition}

\paragraph{The Exponential Map}

\begin{definition}
  \label{definition-exponential-map}
  Given a Lie group \( G \) with Lie algebra \( \mathfrak{g} \), we define a map \( \exp: \mathfrak{g} \to G \), called the \emph{exponential map} of \( G \), as follows: for any \( X \in \mathfrak{g} \), we set
  \[
    \exp X = \gamma(1),
  \]
  where \( \gamma \) is the one-parameter subgroup generated by \( X \), or equivalently the integral curve of \( X \) starting at the identity.
\end{definition}

\begin{proposition}
  \label{proposition-one-parameter-subgroup-under-exp}
  Let \( G \) be a Lie group.
  For any \( X \in \operatorname{Lie}(G), \gamma(s) = \exp s X \) is the one-parameter subgroup of \( G \) generated by \( X \).
\end{proposition}

\begin{proposition}
  \label{proposition-propertie-of-the-exponential-map}
  Let \( G \) be a Lie group and let \( \mathfrak{g} \) be its Lie algebra.
  \begin{enumerate}
    \item The exponential map is a smooth map from \( \mathfrak{g} \) to \( G \).
    \item For any \( X \in \mathfrak{g} \) and \( s, t \in \mathbb{R}, \exp(s + t)X = \exp sX \exp tX \).
    \item For any \( X \in \mathfrak{g}, (\exp X)^{-1} = \exp(-X) \).
    \item For any \( X \in \mathfrak{g} \) and \( n \in \mathbb{Z}, (\exp X)^n = \exp(nX) \).
    \item The differential \( (\dif \exp)_0: T_0 \mathfrak{g} \to T_e G \) is the identity map, under the canonical identifications of both \( T_0 \mathfrak{g} \) and \( T_e G \) with \( \mathfrak{g} \) itself.
    \item The exponential map restricts to a diffeomorphism from some neighborhood of \( 0 \) in \( \mathfrak{g} \) to a neighborhood of \( e \) in \( G \).
    \item If \( H \) is another Lie group, \( \mathfrak{h} \) is its Lie algebra, and \( \Phi: G \to H \) is a Lie group homomorphism, the following diagram commutes:
    \begin{center}
      % https://q.uiver.app/#q=WzAsNCxbMCwwLCJcXG1hdGhmcmFre2d9Il0sWzEsMCwiXFxtYXRoZnJha3tofSJdLFswLDEsIkciXSxbMSwxLCJIIl0sWzAsMiwiXFxleHAiLDJdLFswLDEsIlxcUGhpXyoiXSxbMiwzLCJcXFBoaSIsMl0sWzEsMywiXFxleHAiXV0=
      \begin{tikzcd}
        {\mathfrak{g}} & {\mathfrak{h}} \\
        G & H
        \arrow["{\Phi_*}", from=1-1, to=1-2]
        \arrow["\exp"', from=1-1, to=2-1]
        \arrow["\exp", from=1-2, to=2-2]
        \arrow["\Phi"', from=2-1, to=2-2]
      \end{tikzcd}
    \end{center}
  \item The flow \( \theta \) of a left-invariant vector field \( X \) is given by \( \theta_t = R_{\exp tX} \) which is the right multiplication by \( \exp tX \).
  \end{enumerate}
\end{proposition}

\begin{proposition}
  \label{proposition-Lie-subalgebra-description}
  Let \( G \) be a Lie group, and let \( H \subseteq G \) be a Lie subgroup.
  With \( \operatorname{Lie}(H) \) considered as a subalgebra of \( \operatorname{Lie}(G) \) in the usual way, the exponential map of \( H \) is the restriction to \( \operatorname{Lie}(H) \) of the exponential map of \( G \), and
  \[
    \operatorname{Lie}(H) = \left\lbrace X \in \operatorname{Lie}(G): \exp tX \in H \text{ for all } t \in \mathbb{R} \right\rbrace.
  \]
\end{proposition}

\end{document}
