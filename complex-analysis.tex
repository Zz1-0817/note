\input{preamble}
\begin{document}
\section{Arithmetric property of Complex Number}
% Maybe topology part, and arithmetric part?
For convenience, we assume that \( \alpha, \beta, \gamma, \delta \in \mathbb{R} \).
\paragraph{Addition}
\[
  (\alpha + i\beta) + (\gamma + i \delta) = (\alpha + \gamma) + i(\beta + \delta).
\]
\paragraph{Multiplication}
\[
  (\alpha + i\beta)(\gamma + i\delta) = (\alpha \gamma - \beta \delta) + i(\alpha \delta + \beta \gamma).
\]
\paragraph{Division}
Provided that \( \gamma + i \delta \neq 0 \), then
\[
  \frac{\alpha + i\beta}{\gamma + i\delta} = \frac{(\alpha + i\beta)(\gamma - i\delta)}{(\gamma + i\delta)(\gamma - i\delta)} = \frac{(\alpha \gamma + \beta \delta) + i(\beta \gamma - \alpha \delta)}{\gamma^2 + \delta^2},
\]
in particular,
\[
  \frac{1}{\alpha + i\beta} = \frac{\alpha - i\beta}{\alpha^2 + \beta^2}.
\]
\paragraph{Square Roots}
\[
  \sqrt{\alpha + i\beta} = \pm \left( \sqrt{\frac{\alpha + \sqrt{\alpha^2 + \beta^2}}{2}} \right) + i \frac{\beta}{\left\lvert \beta \right\rvert}\sqrt{\frac{-\alpha + \sqrt{\alpha^2 + \beta^2}}{2}},
\]
since if \( (x + iy)^2 = \alpha + i \beta \) then
\begin{align*}
  x^2 - y^2 = \alpha\\ 2xy = \beta.
\end{align*}
Hence \( (x^2 + y^2)^2 = (x^2 - y^2)^2 + 4x^2y^2 = \alpha^2 + \beta^2 \), which implies \( x^2 + y^2 = \sqrt{\alpha^2 + \beta^2} \).
Now we can solve \( x \) and \( y \).
\paragraph{Conjugation} The transformation which replaces \( \alpha + i \beta \) by \( \alpha - i \beta \) is called \emph{complex conjugation}, and \( \alpha - i \beta \) is the \emph{conjugate} of \( \alpha + i \beta \).
\begin{itemize}
  \item The conjagtion is an \emph{involutory} transformation: this means that \( \overline{\overline{a}} = a \).
  \item \( \operatorname{Re} a = \frac{a + \overline{a}}{2}, \quad \operatorname{Im} a = \frac{a - \overline{a}}{2i} \).
  \item \( \overline{a + b} = \overline{a} + \overline{b}, \overline{ab} = \overline{a} \cdot \overline{b}, (\overline{b / a}) = \overline{b} / \overline{a} \).
    More generally, let \( R(a, b, c, \cdots) \) stand for any rational operation applied to the complex number \( a, b, c, \cdots \). Then
    \[
      \overline{R(a, b, c, \cdots)} = R(\overline{a}, \overline{b}, \overline{c}, \cdots).
    \]
\end{itemize}

\paragraph{Absolute Value}
The product \( a \overline{a} = \alpha^2 + \beta^2 \) is always positive or zero.
Its nonnegative square root is called the \emph{modulus} or \emph{absolute value} of the complex number \( a \); it is denoted by \( \left\lvert a \right\rvert \).
\begin{itemize}
  \item The absolute value of a product is equal to the product of the absolute values of the factors:
    \[
      \left\lvert a_1 a_2 \cdots a_n \right\rvert = \left\lvert a_1 \right\rvert \cdot \left\lvert a_2 \right\rvert \cdots \left\lvert a_n \right\rvert.
    \]
  \item \( \left\lvert a + b \right\rvert^2 = \left\lvert a \right\rvert^2 + \left\lvert b \right\rvert^2 + 2 \operatorname{Re} a \overline{b} \), \(  \left\lvert a - b \right\rvert^2 = \left\lvert a \right\rvert^2 + \left\lvert b \right\rvert^2 - 2 \operatorname{Re} a \overline{b}  \).
    By addition we obtain the identity
    \[
      \left\lvert a + b \right\rvert^2 + \left\lvert a - b \right\rvert^2 = 2 \left( \left\lvert a \right\rvert^2 + \left\lvert b \right\rvert^2 \right).
    \]
\end{itemize}

\paragraph{Inequality}

\paragraph{Geotric Addtion and Multiplication} For \( a_1 = r_1 (\cos \varphi_1 + i \sin \varphi_1) \) and \(  a_2 = r_2 (\cos \varphi_2 + i \sin \varphi_2) \), we have
\[
  a_1 a_2 = r_1 r_2 [\cos (\varphi_1 + \varphi_2) + i \sin (\varphi_1 + \varphi_2)].
\]
The argument of a product is equal to the sum of the arguments of the factors.
\paragraph{The Binomial Equation}
\[
  a^n = r^n(\cos n \varphi + i \sin n \varphi),
\]
for \( n \in \mathbb{Z} \).

To find the \( n \)th root of a complex number \( a \) we have to solve the equation
\[
  z^n = a.
\]
Suppose that \( a \neq 0 \) we write \( a = r(\cos \varphi + i \sin \varphi) \) and \( z = \rho(\cos \theta + i \sin \theta) \).
Then
\[
  z = \sqrt[n]{r}\left[ \cos \left( \frac{\varphi}{n} + k\frac{2 \pi}{n} \right) + i \sin \left( \frac{\varphi}{n} + k \frac{2 \pi}{n} \right) \right],\quad k = 0, 1, \cdots, n - 1.
\]
\paragraph{The Spherical Representation}

% compactication, local compact

\section{Complex Functions}

% One direction: differentian, the other: power series
% For both, polynomials are important

\subsection{Analytic Functions}

%TODO: Cauchy-Riemann equations

\subsection{Polynomials and Rational Functions}

\subsection{Power Series}

\subsection{The Exponential and Trigonometric Functions}

\section{Analytic Functions}

\subsection{Conformal Mappings}

\subsection{Linear Fractional Transformation}

\section{Complex Integration}

\subsection{Path Integration}

\subsection{Cauchy's Theorem and Cauchy's Formula}

\subsection{Taylor Theorems}

\subsection{The General Form of Cauchy's Theorem}

\section{Series and Product Development}

\section{Conformal Mapping and Dirichlet's Problem}

\subsection{The Riemann Mapping Theorem}

\subsection{Harmonic Functions}

\section{Elliptic Functions}

\end{document}
