\documentclass[12pt]{article}

\usepackage{xr-hyper}

\usepackage{geometry}
\usepackage{hyperref}
 
\hypersetup{
  colorlinks = true,
  linkcolor = blue,
  citecolor = red,
  urlcolor = teal
}

\externaldocument[calculus-]{calculus}
\externaldocument[group-]{group}
\externaldocument[field-]{field}
\externaldocument[ode-]{ode}
\externaldocument[functional-analysis-]{functional_analysis}
\externaldocument[pde-]{pde}
\externaldocument[topology-]{topology}
\externaldocument[smooth-manifold-]{smooth_manifold}
\externaldocument[riemannian-manifold-]{riemannian_manifold}
\externaldocument[lie-group-]{lie_group}

\usepackage[leqno]{amsmath}
\usepackage{amssymb}
\usepackage[centercolon]{mathtools}
\usepackage{stmaryrd}
\usepackage{wasysym}
\usepackage{amsthm}
\usepackage{mathrsfs}
\usepackage{bm}

\usepackage{graphicx}
\usepackage{float}

\usepackage{tikz}
\usepackage{tikz-cd}

\geometry{
  paper=a4paper,
  top=3cm,
  inner=2.54cm,
  outer=2.54cm,
  bottom=3cm,
  headheight=6ex,
  headsep=6ex,
  twoside,
  asymmetric
}{\relax}

\usepackage{fancyhdr}
\pagestyle{fancy}
\renewcommand{\sectionmark}[1]{\markright{#1}}
\fancyhf{}
\fancyhead[EC]{\footnotesize{\leftmark}\vspace{1mm}} %页眉部分偶数页显示章
\fancyhead[OC]{\footnotesize{\rightmark}\vspace{1mm}} %页眉部分奇数页显示节
\fancyhead[LE,RO]{{\footnotesize \thepage}\vspace{1mm}} %奇数页右边, 偶数页左边显示页码
\fancyhead[RE,LO]{}
% \fancyfoot[C]{\NTdraftstring}
\renewcommand{\headrulewidth}{0pt} %删去页眉横线
\renewcommand{\footrulewidth}{0pt} %删去页脚横线
\addtolength{\headheight}{0.5pt}

\numberwithin{equation}{subsection}

\theoremstyle{plain}
\newtheorem{theorem}{Theorem}[section]
\newtheorem{lemma}[theorem]{Lemma}
\newtheorem{proposition}[theorem]{Proposition}
\newtheorem{corollary}[theorem]{Corollary}

\theoremstyle{definition}
\newtheorem{definition}[theorem]{Definition}
\newtheorem{example}[theorem]{Example}

\theoremstyle{remark}
\newtheorem{remark}[theorem]{Remark}

\newcommand{\dif}{\mathop{}\!\mathrm{d}}
\renewcommand{\hom}{\operatorname{Hom}}
