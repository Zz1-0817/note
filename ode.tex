\documentclass[12pt]{article}

\usepackage{xr-hyper}

\usepackage{geometry}
\usepackage{hyperref}
 
\hypersetup{
  colorlinks = true,
  linkcolor = blue,
  citecolor = red,
  urlcolor = teal
}

\externaldocument[calculus-]{calculus}
\externaldocument[group-]{group}
\externaldocument[field-]{field}
\externaldocument[ode-]{ode}
\externaldocument[functional-analysis-]{functional_analysis}
\externaldocument[pde-]{pde}
\externaldocument[topology-]{topology}
\externaldocument[smooth-manifold-]{smooth_manifold}
\externaldocument[riemannian-manifold-]{riemannian_manifold}
\externaldocument[lie-group-]{lie_group}

\usepackage[leqno]{amsmath}
\usepackage{amssymb}
\usepackage[centercolon]{mathtools}
\usepackage{stmaryrd}
\usepackage{wasysym}
\usepackage{amsthm}
\usepackage{mathrsfs}
\usepackage{bm}

\usepackage{graphicx}
\usepackage{float}

\usepackage{tikz}
\usepackage{tikz-cd}

\geometry{
  paper=a4paper,
  top=3cm,
  inner=2.54cm,
  outer=2.54cm,
  bottom=3cm,
  headheight=6ex,
  headsep=6ex,
  twoside,
  asymmetric
}{\relax}

\usepackage{fancyhdr}
\pagestyle{fancy}
\renewcommand{\sectionmark}[1]{\markright{#1}}
\fancyhf{}
\fancyhead[EC]{\footnotesize{\leftmark}\vspace{1mm}} %页眉部分偶数页显示章
\fancyhead[OC]{\footnotesize{\rightmark}\vspace{1mm}} %页眉部分奇数页显示节
\fancyhead[LE,RO]{{\footnotesize \thepage}\vspace{1mm}} %奇数页右边, 偶数页左边显示页码
\fancyhead[RE,LO]{}
% \fancyfoot[C]{\NTdraftstring}
\renewcommand{\headrulewidth}{0pt} %删去页眉横线
\renewcommand{\footrulewidth}{0pt} %删去页脚横线
\addtolength{\headheight}{0.5pt}

\numberwithin{equation}{subsection}

\theoremstyle{plain}
\newtheorem{theorem}{Theorem}[section]
\newtheorem{lemma}[theorem]{Lemma}
\newtheorem{proposition}[theorem]{Proposition}
\newtheorem{corollary}[theorem]{Corollary}

\theoremstyle{definition}
\newtheorem{definition}[theorem]{Definition}
\newtheorem{example}[theorem]{Example}

\theoremstyle{remark}
\newtheorem{remark}[theorem]{Remark}

\newcommand{\dif}{\mathop{}\!\mathrm{d}}
\renewcommand{\hom}{\operatorname{Hom}}


\begin{document}

\title{Ordinary Differential Equations}
\label{chapter-ode}


\section{Existence and Uniqueness}
\label{section-existence-and-uniqueness}

\subsection{Basic Concepts}
\label{subsection-basic-concepts}

\paragraph{Phase Space}

\begin{definition}
  \label{definition-deterministic}
  \label{definition-phase-space}
  A process is called \emph{deterministic} if its entire future course and its entire past are uniquely determined by its state at the present time.
  The set of all states of the process is called the \emph{phase space}.
\end{definition}

\begin{definition}
  \label{definition-finite-dimensional}
  A process is called \emph{finite-dimensional} if its phase space is finite-dimensional, i.e., if the number of parameters needed to describe its states is finite.
\end{definition}

\begin{definition}
  \label{definition-differentiable}
  A process is called \emph{differentiable} if its phase space has the structure of a differentiable manifold, and the change of state with time is described by differentiable functions.
\end{definition}


\subsection{First Order Equation}
\label{subsection-existence-and-uniqueness-order-1}


\begin{theorem}
  \label{theorem-existence-and-uniqueness-order-1}
  Given a differential equation
  \begin{equation}
    \dot{x} = f(t, x) \label{equation-single-variable-1-order-ode}
  \end{equation}
  Suppose \( f(t, x) \) is defined on an open set \( \Gamma \) in the plane \( P \) spanned by \( t \) and \( x \), and \( f \) and \( \frac{\partial f}{\partial x} \) is continuous over \( t \) and \( x \), then
  \begin{enumerate}
    \item For any \( (t_0, x_0) \in \Gamma \), there is a solution \( x = \varphi(x) \) of \eqref{equation-single-variable-1-order-ode} satisfying the initial condition
      \[
        \varphi(t_0) = x_0.
      \]
    \item If \( x = \psi(t) \) and \( x = \chi(t) \) are solutions of \eqref{equation-single-variable-1-order-ode} coincides at some \( t_0 \), i.e.
      \[
        \psi(t_0) = \chi(t_0),
      \]
      then the they are equal everywhere.
  \end{enumerate}
\end{theorem}


\section{Constant Coefficients ODE}
\label{section-constant-coefficients-ode}

\subsection{Integrable Cases}
\label{subsection-integrable-cases}

Consider
\begin{equation}
  y' = f(x, y)\label{equation-integrable-case}
\end{equation}
The right-hand side \( f(x, y) \) of the equation is assumed to be defined as a real-valued function on a set \( D \) in the \( xy \)-plane.

\begin{definition}
  \label{definition-solution}
  Let \( J \) be an interval.
  A function: \( y(x): J \to \mathbb{R} \) is called a \emph{solution} to the differential equation \ref{equation-integrable-case} if \( y \) is differentiable in \( J \), the graph of \( y \) is a subset of \( D \), and \ref{equation-integrable-case} holds, i.e., if
  \[
    (x, y(x)) \in D \text{ and } y'(x) = f(x, y(x)) \text{ for all } x \in J.
  \]
\end{definition}

\begin{definition}
  \label{definition-line-element}
  \label{definition-direction-field}
  A numerical triple \( (x, y, p) \) is called a \emph{line element}: \( (x, y) \) gives a point in the plane, and the third component \( p \) gives the slope of a line through the point \( (x, y) \).
  The collection of all line elements of the form \( (x, y, f(x, y)) \) is called a \emph{direction field}.
\end{definition}

\begin{definition}
  \label{definition-initial-condition}
  Let a function \( f(x, y) \), defined on a set \( D \) in the \( (x, y) \)-plane, and a fixed point \( (\xi, \eta) \in D \) be given.
  A function \( y(x) \) is sought that is differentiable in an interval \( J \) with \( \xi \in J \) such that
  \begin{align}
    y'(x) = f(x, y(x)) \in J,\\
    y(\xi) = \eta.
  \end{align}
\end{definition}

\paragraph{Case \( y' = f(x) \)}

Suppose the function \( f(x) \) is continuous in an interval \( J \) and the set \( D \) is a strip \( J \times \mathbb{R} \).
One can prove that all of the solutions can be obtained by translating any particular solution in the direction indicated by the \( y \)-axis.

\paragraph{Case \( y' = g(y) \)}

A formal calculation gives
\[
  \frac{\dif y}{\dif x} = g(y) \iff \frac{\dif y}{g(y)} = \dif x
\]
and h

\end{document}
