\input{preamble}
\begin{document}
\title{Partial Differential Equations}
\label{chapter-partial-differential-equations}

\section{Introduction}
\label{section-introduction}

\subsection{Partial Differential Equations}
\label{subsection-introduction-partial-differential-equations}

\begin{definition}
  \label{definition-partial-differential-equation}
  \label{definition-order-of-differential-equation}
  A partial differential equation is a relation of the following type:
  \begin{equation}
    F(x_1, \cdots, x_n, u, u_{x_1}, \cdots, u_{x_n}, ,u_{x_1 x_2} \cdots) = 0, \label{equation-partial-differential-equation}
  \end{equation}
  where the unknown \( u = u(x_1, \cdots, x_n) \) is a function of \( n \) variables and \( u_{x_j}, \cdots, u_{x_i x_j}, \cdots \) are its partial derivatives.
  The highest order of differentiation occuring in the equation is the  \emph{order} of the equation.
\end{definition}
A first important distinction is between \emph{linear} and \emph{nonlinear} equations.
\begin{definition}
  \label{definition-linear-partial-differential-equation}
  \label{definition-nonlinear-partial-differential-equation}
  Equation \eqref{equation-partial-differential-equation} is \emph{linear} if \( F \) is linear w.r.t. \( u \) and all its derivatives, otherwise it is \emph{nonlinear}.
\end{definition}
A second distinction concerns the types of nonlinearitly.
\begin{definition}
  \label{definition-semilinear-partial-differential-equation}
  \label{definition-quasi-partial-differential-equation}
  \label{definition-fully-nonlinear-partial-differential-equation}
  \begin{enumerate}
    \item \emph{Semilinear} equations when \( F \) is nonlinear only w.r.t. \( u \) but is linear w.t.r. to all its derivatives, with coefficients depending only on \( \mathbf{x} = (x_1, \cdots, x_n) \).
    \item \emph{Quasi-linear} equations when \( F \) is linear w.r.t. the highest order derivatives of \( u \), with coefficients depending only on \( \mathbf{x}, u \) and lower order derivatives.
    \item \emph{Fully nonlinear} equations when \( F \) is nonlinear w.r.t. the highest order derivatives of \( u \).
  \end{enumerate}
\end{definition}

\subsection{Well Posed Problems}
\label{subsection-well-posed-problems}

\section{Linear PDE}
\label{section-linear-pde}

\subsection{Transport Equation}
\label{subsection-transport-equation}

We write \( D u = D_x u = (u_{x_1}, \cdots, u_{x_n}) \) for the gradient of \( u \) w.r.t. spatial variables \( x \).
The \emph{transport equation} with constant coefficients, is the PDE
\begin{equation}
  u_t + b \cdot Du = 0 \text{ in } \mathbb{R}^n \times (0, \infty),
  \label{equation-transport-equation}
\end{equation}
where \( b \) is a fixed vector in \( \mathbb{R}^n \), \( b = (b_1, \cdots, b_n) \), and \( u: \mathbb{R}^n \times [0, \infty) \to \mathbb{R} \) is the unknown.

Define \( z(s) := u(x + sb, t + s)(s \in \mathbb{R}) \), by \eqref{equation-transport-equation}, we then calculate
\begin{equation}
  \dot{z}(s) = Du(x + sb, t + s) \cdot b + u_t(x + sb, t + s) = 0.
  \label{equation-transport-equation-first-attempt}
\end{equation}

\paragraph{Initial-value Problem}

For definition therefore, let us consider the initial-value problem
\begin{equation}
  \begin{cases}
  u_t + b \cdot Du & \text{ in } \mathbb{R}^n \times (0, \infty)\\
  u = g & \text{ on } \mathbb{R}^n \times 0.
  \end{cases}
  \label{equation-transport-equation-with-initial-value}
\end{equation}

From \eqref{equation-transport-equation-first-attempt}, we know that \( u(x - tb, 0) = g(x - tb) \), thus
\begin{equation}
  u(x, t) = g(x - tb) (x \in \mathbb{R}^n, g \geq 0).
  \label{equation-solution-transport-equation-with-initial-value}
\end{equation}

\paragraph{Nonhomogeneous Problem}
Next let us look at the associated nonhomogeneous problem
\begin{equation}
  \begin{cases}
  u_t + b \cdot Du = f & \text{ in } \mathbb{R}^n (0, \infty)\\
  u = g &  \text{ on } \mathbb{R}^n \times 0.
  \end{cases}
  \label{equation-transport-equation-with-nonhomogeneous-initial-value}
\end{equation}
Still, we set \( z(s) := u(x + sb, t + s) \) for \( s \in \mathbb{R} \), then
\[
  \dot{z}(s) = Du(x + sb, t + s) \cdot b + u_t(x + sb, t + s) = f(x + sb, t + s).
\]
Consequently,
\begin{align*}
  u(x, t) - g(x - tb) &= z(0) - z(-t) = \int_{-t}^{0}\dot{z}(s) \dif s\\
                      &= \int_{-t}^{0} f(x + sb, t + s) \dif s\\
                      &= \int_0^{t} f(x + (s - t) b, s) \dif s,
\end{align*}
and so
\begin{equation}
  u(x, t) = g(x - tb) + \int_0^t f(x + (s - t)b , s) \dif s (x \in \mathbb{R}^n, t \geq 0)
  \label{equation-solution-transport-equation-with-nonhomogeneous-initial-value}
\end{equation}
solves the initial-value problem \eqref{equation-transport-equation-with-nonhomogeneous-initial-value}.

\subsection{Laplace's Equation}
\label{subsection-Laplace-equation}

Among the most important of all partial differential equations are undoubtedly \emph{Laplace's equation}
\begin{equation}
  \Delta u = 0
  \label{equation-Laplace-equation}
\end{equation}
and \emph{Poisson's equation}
\begin{equation}
  - \Delta u = f.
  \label{equation-Poisson-equation}
\end{equation}
In both \eqref{equation-Laplace-equation} and \eqref{equation-Poisson-equation}, \( x \in U \) and unknown is \( u: \overline{U} \to \mathbb{R}, u = u(x) \), where \( U \subseteq \mathbb{R}^n \) is a given open set.
In \eqref{equation-Poisson-equation}, the function \( f: U \to \mathbb{R} \) is also given.

\begin{definition}
  \label{definition-harmonic-function}
  A \( C^2 \) function \( u \) satisfying \eqref{equation-Laplace-equation} is called a harmonic function.
\end{definition}

\paragraph{Fundamental Solution}

We attempt to find a solution \( u \) of Laplace's equation \eqref{equation-Laplace-equation} in \( U = \mathbb{R}^n \), having the form
\[
  u(x) = v(r),
\]
where \( r = \left\lvert x \right\rvert = (x_1^2 + \cdots + x^2_n)^{1/2} \) and \( v \) is to be selected so that \( \Delta u = 0 \) holds.
First note for \( i = 1, \cdots, n \) that
\[
  \frac{\partial r}{\partial x_i} = \frac{1}{2}(x^2_1 + \cdots + x^2_n)^{-1/2} 2x_i = \frac{x_i}{r}.
\]
We thus have
\[
  u_{x_i} = v'(r) \frac{x_i}{r},\quad u_{x_i x_i} = v''(r)\frac{x_i^2}{r^2} + v'(r)\left( \frac{1}{r} - \frac{x^2_i}{r^3} \right)
\]
for \( i = 1, \cdots, n \), and so
\[
  \Delta u = v''(r) + \frac{n - 1}{r} v'(r).
\]
Hence \( \Delta u = 0 \) if and only if
\begin{equation}
  v'' + \frac{n - 1}{r}v' = 0.
  \label{equation-radial-equivalent-Laplace-equation}
\end{equation}

If \( v' \neq 0 \), we deduce
\[
  \log (\left\lvert v' \right\rvert)' = \frac{v''}{v'} = \frac{1 - n}{r},
\]
and hence \( v'(r) = \frac{a}{r^{n - 1}} \) for some constant \( a \).
Consequently if \( r > 0 \), we have
\[
  v(r) = \begin{cases}
  b \log r + c & n  = 2\\
  \frac{b}{r^{n - 2}} + c & n \geq 3,
  \end{cases}
\]
where \( b \) and \( c \) are constants.
Hence we define
\begin{definition}
  \label{definition-fundamental-solution-Laplace-equation}
  The function
  \[
    \Phi(x) := \begin{cases}
      -\frac{1}{2 \pi} \log \left\lvert x \right\rvert & n = 2\\
      \frac{1}{n(n - 2)\alpha(n)}\frac{1}{\left\lvert x \right\rvert^{n - 2}} & n \geq 3,
    \end{cases}
  \]
  where \( \alpha(n) \) denotes the volume of the unit ball in \( \mathbb{R}^n \), defined for \( x \in \mathbb{R}^n \), \( x \neq 0 \), is the \emph{fundamental solution} of Laplace's equation.
\end{definition}
\begin{remark}
  \label{remark-fundamental-solution-Laplace-equation}
  Observe that we have the estimates
  \[
    \left\lvert D \Phi(x) \right\rvert \leq \frac{C}{\left\lvert x \right\rvert^{n - 1}},\quad \left\lvert D^2 \Phi(x) \right\rvert \leq \frac{C}{\left\lvert x \right\rvert^n} (x \neq 0)
  \]
  for some constant \( C > 0 \).
\end{remark}

\paragraph{Poisson's Equation}
If we shift the origin to a new point \( y \), the Laplace's equation \eqref{equation-Laplace-equation} is unchanged.
And so \( x \mapsto \Phi(x - y) \) is also harmonic as a function of \( x \), \( x \neq y \).
Now take \( f: \mathbb{R}^n \to \mathbb{R} \), and note that the mapping \( x \mapsto \Phi(x - y) f(y)\quad(x \neq y) \) is harmonic for each point \( y \in \mathbb{R}^n \), and thus so is the sum of finitely many such expressions built for different points \( y \).

What above discussed suggests that the convolution
\begin{equation}
  \begin{aligned}
    u(x) &= \int_{\mathbb{R}^n} \Phi(x - y) f(y) \dif y\\
         &= \begin{cases}
           -\frac{1}{2 \pi} \int_{\mathbb{R}^2} \log (\left\lvert x - y \right\rvert) f(y) \dif y & n = 2\\
           \frac{1}{n(n - 2)\alpha(n)} \int_{\mathbb{R}^n} \frac{f(y)}{\left\lvert x - y \right\rvert^{n - 2}}\dif y & (n \geq 3)
         \end{cases}
  \end{aligned}
  \label{equation-Laplacian-equation-convolution-attempt}
\end{equation}
will solve \eqref{equation-Laplace-equation}.
But this is wrong: \( D^2 \Phi(x - y) \) may be \( \infty \).
%TODO: more details on why this case fails
However, when we assume that \( f \in C^2_c(\mathbb{R}^n) \), \eqref{equation-Laplacian-equation-convolution-attempt} gives a solution for \eqref{equation-Poisson-equation}.

\begin{theorem}
  \label{theorem-solving-Possion-equation}
  Define u by \eqref{equation-Laplacian-equation-convolution-attempt} with \( f \in C^2_c(\mathbb{R}^n) \).
  Then
  \begin{enumerate}
    \item \( u \in C^2 (\mathbb{R}^n) \)
    \item \( - \Delta u = f \) in \( \mathbb{R}^n \).
  \end{enumerate}
  We consequently see that \eqref{equation-Laplacian-equation-convolution-attempt} provides us with a formula for a solution of Poisson's equation \eqref{equation-Poisson-equation} in \( \mathbb{R}^n \).
\end{theorem}
\begin{proof}
  By direct computation(definition), we can find
  \[
    u_{x_i}(x) = \int_{\mathbb{R}^n} \Phi(y) f_{x_i} (x - y) \dif y \text{ and } u_{x_i x_j}(x) = \int_{\mathbb{R}^n} \Phi(y) f_{x_i x_j}(x - y) \dif y,
  \]
  for \( i, j = 1, \cdots, n \).
  Hence \( u \in C^2(\mathbb{R}^n) \).

  In what follows, we use \( C \) to denote some constants if without ambiguity.
  Fix \( \varepsilon > 0 \), then
  \begin{align*}
    \Delta u(x) &= \int_{B(0, \varepsilon)} \Phi(y) \Delta_x f(x - y) \dif y + \int_{\mathbb{R}^n - B(0, \varepsilon)} \Phi(y) \Delta_x f(x - y) \dif y\\
                &=: I_{\varepsilon} + J_{\varepsilon}.
  \end{align*}
  One can check that
  \[
    \left\lvert I_{\varepsilon} \right\rvert \leq
    \begin{cases}
    C \varepsilon^2 \left\lvert \log \varepsilon \right\rvert & n = 2\\
    C \varepsilon^2 & n \geq 3.
    \end{cases}
  \]
  By Green's identity,
  \begin{align*}
    J_{\varepsilon} &= \int_{\mathbb{R}^n - B(0, \varepsilon)}\Phi(y) \Delta_y f(x - y) \dif y\\
                    &= - \int_{\mathbb{R}^n - B(0, \varepsilon)} D \Phi(y) \cdot D_y f(x - y) \dif y + \int_{\partial B(0, \varepsilon)} \Phi(y) \frac{\partial f}{\partial \nu} (x - y) \dif S(y)\\
                    &=: K_{\varepsilon} + L_{\varepsilon},
  \end{align*}
  where \( \nu \) denote the inward pointing unit normal along \( \partial B(0 ,\varepsilon) \).
  We readily check
  \[
    \left\lvert L_{\varepsilon} \right\rvert \leq \left\lVert Df \right\rVert_{L^{\infty}(\mathbb{R}^n)} \int_{\partial B(0, \varepsilon)} \left\lvert \Phi(y) \right\rvert \dif S(y)  \leq
    \begin{cases}
    C \varepsilon \left\lvert \log \varepsilon \right\rvert & n = 2\\
    C \varepsilon & n \geq 3.
    \end{cases}
  \]
  We use the Green identity one again
  \begin{align*}
    K_{\varepsilon} &= \int_{\mathbb{R}^n - B(0, \varepsilon)} \Delta \Phi(y) f(x - y) \dif y - \int_{\partial B(0, \varepsilon)} \frac{\partial \Phi}{\partial \nu}(y) f(x - y) \dif S(y)\\
                    &= - \int_{\partial B(0, \varepsilon)} \frac{\partial \Phi}{\partial \nu}(y) f(x - y)\dif S(y),
  \end{align*}
  since \( \Phi \) is harmonic away from the origin.
  Now \( D \Phi(y) = \frac{-1}{n \alpha(n)} \frac{y}{\left\lvert y \right\rvert^n} \), \( y \neq 0 \), and \( \nu = \frac{-y}{\left\lvert y \right\rvert} = -\frac{y}{\varepsilon} \) on \( \partial B(0, \varepsilon) \).
  Consequently, \( \frac{\partial \Phi}{\partial \nu}(y) = \nu \cdot D \Phi(y) = \frac{1}{n \alpha(n) \varepsilon^{n - 1}} \) on \( \partial B(0, \varepsilon) \).
  Since \( n \alpha(n) \varepsilon^{n - 1} \) is the surface area of the sphere \( \partial B(0, \varepsilon) \), we have
  \begin{align*}
    K_{\varepsilon} &= -\frac{1}{n \alpha(n)\varepsilon^{n - 1}}\int_{\partial B(0, \varepsilon)} f(x - y)\dif S(y)\\
                    &= - \int_{\partial B(x, \varepsilon)} f(y) \dif S(y)\to - f(x) \text{ as } \varepsilon \to 0.
  \end{align*}
\end{proof}

\paragraph{Mean-Value Formulas}

\begin{theorem}[Mean-value formula for Laplace equation]
  \label{theorem-mean-value-formula-for-Laplace-equation}
  Let \( U \subseteq \mathbb{R}^n \) be an open set, and \( B(x, r) \subseteq U \).
  If \( u \in C^2(U) \) is harmonic, then
  \begin{equation}
    u(x) = \frac{1}{n \alpha(n) r^{n - 1}}\int_{\partial B(x, r)} u \dif S = \frac{1}{\alpha(n)r^n} \int_{B(x, r)} u \dif y
    \label{equation-mean-value-formula-for-Laplace-equation}
  \end{equation}
  for each ball \( B(x, r) \subseteq U \).
\end{theorem}
\begin{proof}
  Set
  \[
    \phi(r) := \frac{1}{n \alpha(n) r^{n - 1}}\int_{\partial B(x, r)} u(y) \dif S(y) = \frac{1}{n \alpha(n) r^{n - 1}}\int_{\partial B(0, 1)} u(x + rz) \dif S(z).
  \]
  Then
  \[
    \phi'(r) = \frac{1}{n \alpha(n) r^{n - 1}}\int_{\partial B(0, 1)} D u(x + rz) \cdot z \dif S(z),
  \]
  and consequently, using Green's formulas, we compute
  \begin{align*}
    \phi'(r) &= \frac{1}{n\alpha(n)r^{n - 1}}\int_{\partial B(x, r)} D u(y) \cdot \frac{y - x}{r} \dif S(y)\\
             &= \frac{1}{n\alpha(n)r^{n - 1}}\int_{\partial B(x, r)}\frac{\partial u}{\partial \nu}\dif S(y)\\
             &= \frac{r}{n} \frac{1}{n\alpha(n)r^{n - 1}} \int_{B(x, r)} \Delta u(y) \dif y = 0
  \end{align*}
  Hence \( \phi \) is constant, and so
  \[
    \phi(r) = \lim\limits_{t \to 0} \phi(t) = \frac{1}{n\alpha(n)r^{n - 1}}\lim\limits_{t \to 0}\int_{\partial B(x, t)} u(y) \dif S(y) = u(x).
  \]
  Then
  \begin{align*}
    \int_{B(x, r)} u \dif y &= \int_0^r \left( \int_{\partial B(x, s)} u \dif S \right)\dif s\\
                            &= u(x) \int_0^r n \alpha(n) s^{n - 1} \dif s = \alpha(n) r^n u(x).
  \end{align*}
\end{proof}
Its inverse is also true.
\begin{theorem}
  \label{theorem-converse-to-mean-value-property}
  If \( u \in C^2(U) \) satisfies
  \[
    u(x) = \frac{1}{n \alpha(n) r^{n - 1}} \int_{\partial B(x, r)} u \dif S
  \]
  for each ball \( B(x, r) \subseteq U \), then \( u \) is harmonic.
\end{theorem}
\begin{proof}
  If \( \Delta u \not\equiv 0 \), there exists some ball \( B(x, r) \subseteq U \) such that, say, \( \Delta u > 0 \) within \( B(x, r) \).
  But
  \[
    0 = \phi'(r) = \frac{1}{n \alpha(n) r^{n - 1}}\frac{r}{n}\Delta u(y) \dif y > 0,
  \]
  a contradiction.
\end{proof}

\paragraph{Properties of Harmonic Functions}

\begin{theorem}[Strong Maximum Principle]
  \label{theorem-strong-maximum-principle}
  Suppose \( u \in C^2(U) \cap C(\overline{U}) \) is harmonic within \( U \).
  \begin{enumerate}
    \item Then
      \[
        \max_{\overline{U}} u = \max_{\partial U} u.
      \]
    \item Furthermore, if \( U \) is connected and there exists a point \( x_0 \in U \) such that
      \[
        u(x_0) = \max_{\overline{U}} u,
      \]
      then \( u \) is constant within \( U \).
  \end{enumerate}
  Replacing \( u \) by \( -u \), we recover also similar assertions with ``min'' replacing ``max''.
\end{theorem}
\begin{proof}
  Suppose there exists a point \( x_0 \in U \) with \( u(x_0) = \max_{\overline{U}} u =: M \).
  Then for \( 0 < r < \operatorname{dist}(x_0, \partial U) \), the mean-value property asserts
  \[
    M = u(x_0) = \frac{1}{\alpha(n)r^n} \int_{B(x_0, r)} u \dif y \leq M.
  \]
  As equality holds only if \( u \equiv M \) within \( B(x_0, r) \).
  Hence, (2) is proved.
  And (1) follows from (2).
\end{proof}

\begin{corollary}
  \label{corollary-strong-maximum-principle}
  If \( U \) is connected and \( u \in C^2(U) \cap C(\overline{U}) \) satisfies
  \[
    \begin{cases}
    \Delta u = 0 & \text{ in } U\\
    u = g & \text{ on } \partial U,
    \end{cases}
  \]
  where \( g \geq 0 \), then \( u \) is positive everywhere in \( U \) if \( g \) is positive somewhere on \( \partial U \).
\end{corollary}

\begin{theorem}[Uniqueness]
  \label{theorem-uniqueness-Laplace-equation}
  Let \( g \in C(\partial U) \), \( f \in C(U) \).
  Then there exists at most one solution \( u \in C^2(U) \cap C(\overline{U}) \) of the boundary-value problem
  \[
    \begin{cases}
    - \Delta u = f & \text{ in } U\\
    u = g & \text{ on } \partial U.
    \end{cases}
  \]
\end{theorem}
\begin{proof}
  Apply \ref{theorem-strong-maximum-principle}.
\end{proof}

\begin{theorem}[Smoothness]
  \label{theorem-smoothness-Laplace-equation}
  If \( u \in C(U) \) satisfies the mean-value property \eqref{equation-mean-value-formula-for-Laplace-equation} for each ball \( B(x, r) \subseteq U \), then
  \[
    u \in C^{\infty}(U).
  \]
\end{theorem}

\end{document}
