\input{preamble}

\begin{document}

\title{Group}
\label{chapter-group}

\tableofcontents

\section{Basic Definitions}
\label{section-basic-definitions}

\subsection{Definitions}
\label{subsection-definitions}

\begin{definition}
  \label{definition-group}
  A \emph{group} is a set \( G \) together with a binary operation \( * \):
  \[
    (a, b) \mapsto a * b: G \times G \to G
  \]
  satisfying the following
  \begin{enumerate}
    \item for all \( a, b, c \in G \),
      \[
        (a * b) * c = a * (b * c).
      \]
    \item there exists an element(called \emph{neutral element}) \( e \in G \) such that
      \[
        a * e = a = e * a
      \]
      for all \( a \in G \).
    \item for each \( a \in G \), there exists an \( a' \in G \)(called \emph{inverse} of \( a \), and denoted it \( a^{-1} \)) such that
      \[
         a * a' = e = a' * a.
      \]
  \end{enumerate}
\end{definition}
\begin{remark}
  \label{remark-alternate-group-definition}
  The group conditions 2. and 3. can be replaced by the following weaker conditions:
  \begin{itemize}
    \item there exists an \( e \) such that \( e * a = a \) for all \( a \).
    \item for each \( a \in G \), there exits an \( a' \in G \) such that \( a' * a = e \).
  \end{itemize}
\end{remark}

\begin{definition}
  \label{definition-magma}
  \label{definition-semigroup}
  \label{definition-monoid}
  \begin{enumerate}
    \item A set \( S \) together with a binary operation \( (a, b) \mapsto a \cdot b: S \times S \to S \) is called a \emph{magma}.
    \item When the binary operation is associative, \( (S, \cdot) \) is called a \emph{semigroup}.
    \item A semigroup with a neutral element is called a \emph{monoid}.
  \end{enumerate}
\end{definition}

\begin{definition}
  \label{definition-order}
  \label{definition-p-group}
  \begin{enumerate}
    \item The \emph{order} \( \left\vert G \right\vert \) of a group \( G \) is its cardinality.
    \item A finite group whose order is a power of a prime \( p \) is called a \emph{\( p \)-group}.
    \item For an element \( a \) of a group \( G \), define
      \[
        a^n = \begin{cases}
          \underbrace{a a \cdots a}_{n \text{copies}} & n > 0\\
          e & n = 0\\ 
          \underbrace{a^{-1} a^{-1} \cdots a^{-1}}_{n \text{copies}} & n < 0
        \end{cases}
      \]
    \item The set \( \left\lbrace n \in \mathbb{Z}: a^n = e \right\rbrace \) is an ideal in \( \mathbb{Z} \), and so equals \( m \mathbb{Z} \) for some integer \( m \geq 0 \).
      \begin{itemize}
        \item When \( m = 0, a^n = e  \) unless \( n = 0 \), and \( a \) is said to have \emph{infinite order}.
        \item When \( m \neq 0 \), it is the smallest integer \( m > 0 \) such that \( a^m = e \), and \( a \) is said to have \emph{finite order} \( m \).
      \end{itemize}
  \end{enumerate}
\end{definition}

\begin{definition}
  \label{definition-direct-product}
  When \( G \) and \( H \) are  groups, we can construct a new group \( G \times H \), called the \emph{direct product} of \( G \) and \( H \).
  As a set, it is the Cartesian product of \( G \) and \( H \), and multiplication is defined by
  \[
    (g, h)(g', h') := (gg', hh').
  \]
\end{definition}

\begin{definition}
  \label{definition-commutative-group}
  A group \( G \) is \emph{commutative}(or \emph{abelian}) if
  \[
    ab = ba, \text{ all } a, b \in G.
  \]
\end{definition}

\subsection{Subgroups}
\label{subsection-subgroups}


\begin{definition}
  \label{definition-subgroup}
  Let \( S \) be a nonempty subset of a group \( G \).
  If
  \begin{enumerate}
    \item \( a, b \in S \implies ab \in S \), and
    \item \( a \in S \implies a^{-1} \in S \),
  \end{enumerate}
  then the binary operation on \( G \) makes \( S \) into a group, called a \emph{subgroup} of \( G \).
\end{definition}
\begin{remark}
  \label{remark-subgroup-conditions}
  When \( S \) is finite, condition 1. implies 2..
\end{remark}

\begin{definition}
  \label{definition-centre}
  The \emph{centre} of a group \( G \) is the subset
  \[
    Z(G) := \left\lbrace g \in G: gx = xg \text{ for all } x \in G \right\rbrace.
  \]
  It is a subgroup of \( G \).
\end{definition}

\begin{definition}
  \label{definition-torsion-subgroup}
  In a commutative group \( G \), the elements of finite order form a subgroup \( G_{\text{tor}} \) of \( G \), called the \emph{torsion subgroup}.
\end{definition}


\begin{proposition}
  \label{proposition-intersection-of-subgroups}
  An intersection of subgroups of \( G \) is a subgroup of \( G \).
\end{proposition}

\begin{example}
  \label{example-subgroup-product-not-subgroup}
  However, the product of subgroup need NOT to be a subgroup, consider \( G = S_3, U = (13), V = (12) \).
\end{example}

\begin{proposition}
  \label{proposition-subset-generate-subgroup}
  For any subset \( X \) of a group \( G \), there is a smalleset subgroup of \( G \) containing \( X \).
  It consists of all finite products of elements of \( X \) and their inverses.
\end{proposition}

\begin{definition}
  \label{definition-generate}
  The subgroup \( S \) in the above proposition is denoted \( \left\langle X \right\rangle \), and is called the \emph{subgroup generated} by \( X \).
  In particular, \( \left\langle \emptyset \right\rangle = \left\lbrace e \right\rbrace \).
  We say that \( X \) \emph{generates} if \( G = \left\langle X \right\rangle \).
\end{definition}


\begin{definition}
  \label{definition-coset}
  For a subset \( S \) of a group \( G \) and an element \( a \) of \( G \), we let
  \[
    aS = \left\lbrace as: s \in S \right\rbrace,\quad Sa = \left\lbrace sa: s \in S \right\rbrace.
  \]
  When \( H \) is a subgroup of \( G \), the sets of the form \( aH \) are called the \emph{left cosets} of \( H \) in \( G \), and the sets of the form \( Ha \) are called the \emph{right cosets} of \( H \) in \( G \).
\end{definition}

\begin{proposition}
  \label{proposition-coset-properties}
  Let \( H \) be a subgroup of a group \( G \).
  \begin{enumerate}
    \item An element \( a \) of \( G \) lies in a left coset \( C \) of \( H \iff C = aH \).
    \item Two left cosets are either disjoint or equal.
    \item \( a H = b H \iff a^{-1}b \in H \).
    \item Any two left coset have the same number of elements.
  \end{enumerate}
\end{proposition}

\begin{definition}
  \label{definition-index}
  The \emph{index} \( (G: H) \) of \( H \) in \( G \) is defined to be the number of left cosets of \( H \) in \( G \).
  In particular, \( (G: 1) \) is the order of \( G \).
\end{definition}

\begin{theorem}[Lagrange]
  \label{theorem-Lagrange}
  If \( G \) is finite, then
  \[
    (G : 1) = (G : H)(H : 1).
  \]
  In particular, the order of every subgroup of a finite group divides the order of the group.
\end{theorem}
\begin{remark}
  \label{remark-Lagrange-converses}
  Lagrange's theorem has partial converses:
  \begin{enumerate}
    \item (Cauchy's theorem)if a prime \( p \) divides \( m = (G: 1) \), then \( G \) has an element of order \( p \).
    \item (Sylow's theorem)if a prime power \( p^n \) divides \( m \), then \( G \) has a subgroup of order \( p^n \).
  \end{enumerate}
  However, Klein \( 4 \)-group \( C_2 \times C_2 \) has no element of order \( 4 \); \( A_4 \) has order \( 12 \), but has no subgroup of order \( 6 \).
\end{remark}

\begin{corollary}
  \label{corollary-element-order-divide-group-order}
  The order of each element of a finite group divides the order of the group.
\end{corollary}

\begin{proposition}
  \label{proposition-generalization-Lagrange}
  For any subgroups \( H \supseteq K \) of \( G \), 
  \[
    (G: K) = (G: H)(H: K).
  \]
  (meaning either both are infinite or both are finite and equal).
\end{proposition}
\begin{proof}
  Write \( G = \sqcup_{i \in I} g_i H \) and \( H = \sqcup_{j \in J} h_jK \), then \( G = \sqcup_{i, j \in I \times J} g_i h_j K \).
\end{proof}

\subsection{Normal Subgroups}
\label{subsection-normal-subgroups}


\begin{definition}
  \label{definition-normal-subgroup}
  A subgroup \( N \) of \( G \) is \emph{normal}, denoted \( N \triangleleft G \), if \( gNg^{-1} = N \) for all \( g \in G \).
\end{definition}

\begin{remark}
  \label{remark-verify-subgroup-normal}
  To show that \( N \) is normal, it suffices to check that \( g N g^{-1} \subseteq N \) for all \( g \), because it implies that \( N \subseteq g^{-1} N g \), and rewriting this with gives that \( N \subseteq g N g^{-1} \) for all \( g \).
  However, the next example shows that there can exist a subgroup \( N \) of a group \( G \) and an element \( g \) of \( G \) such that \( g N g^{-1} \subseteq N \) but \( g N g^{-1} \neq N \).
\end{remark}

\begin{example}
  \label{example-conjugated-subgroup-contained-but-not-equal}
  Let \( G = \operatorname{GL}_2(\mathbb{Q}) \), and let \( H = \left\lbrace \begin{pmatrix}
      1 &n\\0 &1
  \end{pmatrix}: n \in \mathbb{Z} \right\rbrace \).
  Then \( H \) is a subgroup of \( G \) and \( H \simeq \mathbb{Z} \).
  Let \( g = \begin{pmatrix}
    5 & 0\\0 & 1
  \end{pmatrix} \).
  Then
  \[
    g \begin{pmatrix}
      1 &n \\0 &1
    \end{pmatrix} g^{-1} = \begin{pmatrix}
      1 &5n\\ 0 &1
    \end{pmatrix}
  \]
  Hence \( g H g^{-1} \subsetneq H \).
\end{example}

\begin{proposition}
  \label{proposition-left-right-coset-of-normal-subgroup}
  A subgroup \( N \) of \( G \) is normal \( \iff \) every left coset of \( N \) in \( G \) is also a right coset, in which case, \( gN = N g \) for all \( g \in G \).
\end{proposition}

\begin{proposition}
  \label{proposition-index-2-subgroup-is-normal}
  Every subgroup of index two is normal.
\end{proposition}
\begin{proof}
  Indeed, let \( g \in G \backslash H \).
  Then \( G = H \sqcup gH \).
  It implies that \( gH \) and \( Hg \) are the complements of \( H \) in \( G \), and hence they are equal.
\end{proof}

\begin{example}
  \label{example-normal-subgroup-in-dihedral-group}
  Consider the dihedral group
  \[
    D_n = \left\lbrace e, r, \cdots, r^{n - 1}, s, \cdots,r^{n - 1}s \right\rbrace.
  \]
  Then \( C_n = \left\lbrace e, r, \cdots, r^{n - 1} \right\rbrace \) has index \( 2 \) and hence is normal.
  For \( n \geq 2 \), the subgroup \( \left\lbrace e, s \right\rbrace \) is not normal because \( r^{-1}sr = r^{n - 2}s \notin \left\lbrace e, s \right\rbrace \).
\end{example}

Similar to \ref{proposition-intersection-of-subgroups}, we have
\begin{proposition}
  \label{proposition-intersection-of-normal-subgroups}
  An intersection of normal subgroups of a group is again a normal subgroup.
  Therefore, we can define the \emph{normal subgroup generated by a subset} \( X \) of a group \( G \) to be the intersection of the normal subgroups containing \( X \).
\end{proposition}

In \ref{example-subgroup-product-not-subgroup}, we found that the product of subgroups need not to be a subgroup.
If we enhance the condition to normal subgroup, then the statement will be true.

\begin{theorem}
  \label{theorem-normal-and-subgroups}
  If \( H \) and \( N \) are subgroups of \( G \) and \( N \) is normal, then \( HN \) is a subgroup of \( G \).
  If \( H \) is also normal, then \( HN \) is a normal subgroup of \( G \).
\end{theorem}

Like \ref{proposition-subset-generate-subgroup}, we want to generate a normal subgroup by a subset.
In order to do this, we need more preparations.

\begin{definition}
  \label{definition-normal-subset}
  We say that a subset \( X \) of a group \( G \) is \emph{normal} if \( gXg^{-1} \subseteq X \) for all \( g \in G \).
\end{definition}

\begin{lemma}
  \label{lemma-set-generate-normal-subgroup}
  \begin{enumerate}
    \item If \( X \) is normal, then the subgroup \( \left\langle X \right\rangle \) is normal.
    \item For any subset \( X \) of \( G \), the subset \( \bigcup_{g \in G} g X g^{-1} \) is normal, and it is the smallest normal set containing \( X \).
  \end{enumerate}
\end{lemma}

\begin{proposition}
  \label{proposition-set-generate-normal-subgroup}
  The normal subgroup generated by a subset \( X \) of \( G \) is \( \left\langle \bigcup_{g \in G} g X g^{-1} \right\rangle \).
\end{proposition}

We can give another result about the largest normal subgroup contained in a given subgroup.
\begin{lemma}
  \label{lemma-largest-normal-subgroup-in-subgroup}
  For any subgroup \( H \) of a group \( G \), \( \bigcap_{g \in G}g H g^{-1} \) is the largest normal subgroup of \( G \) contained in \( H \).
\end{lemma}

\subsection{Examples}
\label{subsection-examples}

\begin{definition}
  \label{definition-cyclic-group}
  A group is said to be \emph{cyclic} if it is generated by a single element, i.e. if \( G = \left\langle r \right\rangle \) for some \( r \in G \).
\end{definition}

If \( r \) has finite order \( n \), then
\[
  G = \left\lbrace e, r, r^2, \cdots, r^{n - 1} \right\rbrace \simeq C_n,\quad r^i \leftrightarrow i \mod{n},
\]
and \( G \) can be thought of as the group of rotational symmetries about the centre of a regular polygon with \( n \)-sides.
If \( r \) has infinitely order, then
\[
  G = \left\lbrace \cdots, r^{-i}, \cdots, r^{-1}, \cdots, e, r, r, \cdots, r^i, \cdots \right\rbrace \simeq C_\infty,\quad r^i \leftrightarrow i.
\]
Thus, up to isomorphism, there is exactly one cyclic group of order \( n \)  for each \( n \leq \infty \).

\begin{proposition}
  \label{proposition-generator-of-cyclic-group}
  Let \( G = \left\langle a \right\rangle \) be a cyclic group of order \( n \).
  Then the generators of \( G \) are exactly the elements \( a^m \) with \( \gcd(m, n) = 1 \), where \( m \geq 1 \).
\end{proposition}

\begin{example}
  \label{example-units-of-finite-field-cyclic}
  The units of finite field form a cyclic group.
  In particular, \( \mathbb{F}_p^\times \) is cyclic for some prime.
\end{example}
\begin{proof}
  \ref{corollary-commutative-group-cyclic-if-condition}.
\end{proof}

\subsection{Homomorphisms}
\label{subsection-homomorphisms}


\begin{definition}
  \label{definition-homomorphism}
  \label{definition-isomorphism}
  A \emph{homomorphism} from a group \( G \) to a second \( G' \) is a map \( \alpha: G \to G' \) such that \( \alpha(ab) = \alpha(a) \alpha(b) \) for all \( a, b \in G \).
  An \emph{isomorphism} is a bijective homomorphism.
\end{definition}

\begin{theorem}[Cayley]
  \label{theorem-Cayley}
  There is a canonical injective homomorphism
  \[
    \alpha: G \to \operatorname{Sym}(G).
  \]
\end{theorem}
\begin{corollary}
  \label{corollary-realize-finite-group-as-permutation}
  A finite group of order \( n \) can be realized as a subgroup of \( S_n \).
\end{corollary}


\begin{definition}
  \label{definition-kernel}
  The \emph{kernel} of a homomorphism \( \alpha: G\to G" \) is
  \[
    \ker (\alpha) = \left\lbrace g \in G: \alpha(g) = e \right\rbrace.
  \]
\end{definition}

\begin{proposition}
  \label{proposition-kernel-properties}
  \begin{enumerate}
    \item \( \alpha \) is injective \( \iff \ker (\alpha) = \left\lbrace e \right\rbrace \).
    \item The kernel of a homomorphism is a normal subgroup.
  \end{enumerate}
\end{proposition}

\begin{example}
  \label{example-determinant-kernel}
  The kernel of the homomorphism \( \det: \operatorname{GL}_n(F) \to F^\times \) is the group of \( n \times n \) matrics with determinant \( 1 \), this group \( \operatorname{SL}_n(F) \) is called the \emph{special linear group} of \emph{degree} \( n \).
\end{example}

\begin{proposition}
  \label{proposition-every-normal-subgroup-as-kernel}
  \label{definition-quotient-group}
  Every normal subgroup occurs as the kernel of a homomorphism.
  More precisely, if \( N \) is a normal subgroup of \( G \), then there is a unique group structure on the set \( G / N \) of cosets of \( N \) in \( G \) for which the natrual map
  \[
    a \mapsto [a] : G \to G / N
  \]
  is a homomorphism.
  The group \( G / N \) is called the \emph{quotient} of \( G  \) by \( N \).
\end{proposition}

\begin{proposition}
  \label{proposition-kernel-universal-property}
  The map \( a \mapsto a N: G \to G / N \) has the following universal property:
  for any homomorphism \( \alpha: G \to G' \) of groups such that \( \alpha(N) = \left\lbrace e \right\rbrace \), there exists a unique homomorphism \( G / N \to G' \) making the diagram commute.
\begin{center}
  \begin{tikzcd}
    G & { G/ N} \\
    & {G'}
    \arrow[from=1-1, to=1-2]
    \arrow[from=1-1, to=2-2]
    \arrow[dashed, from=1-2, to=2-2]
  \end{tikzcd}
\end{center}
\end{proposition}


\begin{theorem}
  \label{theorem-decomposition-of-homomorphism}
  For any homomorphism \( \alpha: G \to G' \) of groups, the kernel \( N \) of \( \alpha \) is a normal subgroup of \( G \), the image \( I \) of \( \alpha \) is a subgroup of \( G' \), and \( \alpha \) factors in a natural way into the composite of a surjection, an isomorphism, and an injection:
\begin{center}
  \begin{tikzcd}
    G & {G/N} & I & {G'}
    \arrow[from=1-1, to=1-2]
    \arrow[from=1-2, to=1-3]
    \arrow[from=1-3, to=1-4]
  \end{tikzcd}
\end{center}
\end{theorem}

\begin{theorem}
  \label{theorem-isomorphism}
  Let \( H \)b e a subgroup of \( G \) and \( N \) a normal subgroup of \( G \).
  Then \( HN \) is a subgroup of \( G \), \( H \cap N \) is a normal subgroup of \( H \), and the map
  \[
    h(H \cap N) \mapsto hH: H / H \cap N \to HN / N
  \]
  is an isomorphism.
\end{theorem}

\begin{theorem}
  \label{theorem-correspondence}
  Let \( \alpha: G \twoheadrightarrow \widetilde{G} \) be a surjective homomorphism, and let \( N = \ker(\alpha) \).
  Then there is a one-to-one correspondence
  \[
    \left\lbrace \text{subgroup of } G \text{ containing }N \right\rbrace \mathop{\leftrightarrow}\limits^{1:1} \left\lbrace \text{subgroups of } \widetilde{G} \right\rbrace
  \]
  under which a subgroup \( H \) of \( G \) containing \( N \) corresponds to \( \widetilde{H} \) of \( \widetilde{G} \) coresponds to \( H = \alpha^{-1}(H) \).
  Moreover, if \( H \leftrightarrow \widetilde{H} \) and \( H' \leftrightarrow \widetilde{H}' \), then
  \begin{enumerate}
    \item \( \widetilde{H} \subseteq \widetilde{H}' \iff H \subseteq H' \), in which case \( (\widetilde{H}': \widetilde{H}) = (H' : H) \);
    \item \( \widetilde{H} \) is normal in \( \widetilde{G} \iff H \) is normal in \( G \), in which case, \( \alpha \) induces an isomorphism
      \[
        G / H \simeq \widetilde{G} / \widetilde{H}.
      \]
  \end{enumerate}
\end{theorem}

\begin{corollary}
  \label{corollary-correspondence}
  Let \( N \) be a normal subgroup of \( G \).
  \begin{enumerate}
    \item Then there is a one-to-one correspondence between the set of subgroups of \( G \) containing \( N \) and the set of subgroups of \( G / N \)
    \item Moreover, \( H \) is normal in \( G \iff H / N \) is normal in \( G / N \), in which case the homomorphism \( g \mapsto gN: G \to G / N \) induces an isomorphism
      \[
        G / H \simeq (G / N) / (H / N).
      \]
  \end{enumerate}
\end{corollary}

\subsection{Direct Products}
\label{subsection-direct-products}

We now give a generalization of \ref{definition-direct-product}:
\begin{definition}
  \label{definition-generalization-direct-product}
  Let \( G \) be a group, and let \( H_1, \cdots, H_k \) be subgroups of \( G \).
  We says that \( G \) is a \emph{direct product} of the subgroups \( H_i \) if the map
  \[
    H_1 \times H_2 \times \cdots \times H_k \to G: (h_1, h_2, \cdots, h_k) \mapsto h_1h_2\cdots h_k
  \]
  is an isomorphism of groups.
\end{definition}
\begin{remark}
  \label{remark-element-in-direct-product}
  This means that each element \( g \) of \( G \) can be written uniquely in the form \( g = h_1h_2 \cdots h_k, h_i \in H_i \), and that if \( g = h_1 h_2 \cdots h_k \) and \( g' = h'_1 h'_2 \cdots h'_k \), then
  \[
    g g' = (h_1 h_1')(h_2 h_2') \cdots (h_k h_k').
  \]
\end{remark}

\begin{proposition}
  \label{proposition-iff-conditions-of-direct-product}
  A group \( G \) is a direct product of subgroups \( H_1, H_2 \) if and only if
  \begin{enumerate}
    \item \( G = H_1 H_2 \),
    \item \( H_1 \cap H_2 = \left\lbrace e \right\rbrace \),
    \item One of the followings holds:
      \begin{enumerate}
        \item every element of \( H_1 \) commutes with every element of \( H_2 \).
        \item \( H_1 \) and \( H_2 \) are both normal in \( G \). 
      \end{enumerate}
  \end{enumerate}
\end{proposition}

\begin{proposition}
  \label{proposition-iff-conditions-of-generalized-direct-product}
  A group \( G \) is a direct product of subgroups \( H_1, H_2, \cdots, H_k \) if and only if
  \begin{enumerate}
    \item \( G = H_1 H_2 \cdots H_k \),
    \item for each \( j \), \( H_j \cap (H_1 \cdots H_{j - 1} H_{j + 1} \cdots H_k) = \left\lbrace e \right\rbrace \),
    \item each of \( H_1, H_2, \cdots, H_k \) is normal in \( G \).
  \end{enumerate}
\end{proposition}

\subsection{Commutative Groups}
\label{subsection-commutative-groups}
In this subsection, let \( M \) be a commutative group, written additively.
The subgroup \( \left\langle x_1, \cdots, x_k \right\rangle \) of \( M \) generated by the elements \( x_1, \cdots, x_k \) consists of the sums \( \sum m_i x_i, m_i \in \mathbb{Z} \).

\begin{definition}
  \label{definition-basis}
  A subset \( \left\lbrace x_1, \cdots, x_k \right\rbrace \) of \( M \) is a \emph{basis} for \( M \) if it generates \( M \) and
  \[
    m_1 x_1 + \cdots + m_k x_k = 0,\quad m_i \in \mathbb{Z} \implies m_i x_i = 0 \text{ for every } i
  \]
  then
  \[
    M = \left\langle x_1 \right\rangle \oplus \cdots \oplus \left\langle x_k \right\rangle.
  \]
\end{definition}

\begin{lemma}
  \label{lemma-change-basis}
  Let \( x_1, \cdots, x_k \) generate \( M \).
  For any \( c_1, \cdots, c_k \in \mathbb{N} \) with \( \gcd (c_1, \cdots, c_k) = 1 \), there exists generators \( y_1, \cdots, y_k \) for \( M \) such that \( y_1 = c_1 x_1 + \cdots + c_k x_k \).
\end{lemma}
\begin{proof}
  We argue by induction on \( s = c_1 + \cdots + c_k \).
  The lemma certainly holds if \( s = 1 \), and so we assume \( s > 1 \).
  Then, at least two \( c_i \) are nonzero, say, \( c_1 \geq c_2 > 0 \).
  Now
  \begin{itemize}
    \item \( \left\lbrace x_1, x_2 + x_1, x_3, \cdots, x_k \right\rbrace \) generates \( M \).
    \item \( \gcd(c_1 - c_2, c_2, c_3, \cdots, c_k) = 1 \),
    \item \( (c_1 - c_2) + c_2 + \cdots + c_k < s \),
  \end{itemize}
  and so, by induction, there exist generators \( y_1, \cdots, y_k \) for \( M \) with
  \begin{align*}
    y_1 &= (c_1 - c_2) x_1 + c_2(x_1 + x_2) + c_3 x_3 + \cdots + c_k x_k\\
        &= c_1 x_1 + \cdots + c_k x_k.
  \end{align*}
\end{proof}

\begin{theorem}
  \label{theorem-finite-generated-group-has-basis}
  Every finitely generated commutative group \( M \) has a basis;
  hence it is a finite direct sum of cyclic group.
\end{theorem}
\begin{proof}
  We argue by induction on the number of generators of \( M \).
  If \( M \) can be generated by one element, the statement is trivial, and so we may assume that it requires at least \( k > 1 \) generators.
  Among the generating sets \( \left\lbrace x_1, \cdots, x_k \right\rbrace \) for \( M \) with \( k \) elements there is one for which the order of \( x_1 \) is the smallest possible.

  We claim that \( M \) is then the direct sum of \( \left\langle x_1 \right\rangle \) and \( \left\langle x_2, \cdots, x_k \right\rangle \), then the proof is completed by induction.
    If \( M \) is not the direct sum of \( \left\langle x_1 \right\rangle \) and \( \left\langle x_2, \cdots, x_k \right\rangle \), then there exists a relation
    \[
      m_1 x_1 + m_2 x_2 + \cdots + m_k x_k = 0
    \]
    with \( m_1 x_1 \neq 0 \).
    After possibly changing the sign of some of the \( x_i \), we may suppose that \( m_1, \cdots, m_k \in \mathbb{Z}_{\geq 0} \) and \( m_1 < \operatorname{order}(x_1) \).
    Let \( d = \gcd (m_1, \cdots, m_k) > 0 \) and let \( c_i = m_i / d \).
    According to the \ref{lemma-change-basis}, there exists a generating set \( y_1, \cdots, y_k \) such that \( y_1 = c_1 x_1 + \cdots + c_k x_k \).
    But
    \[
      d y_1 = m_1 x_1 + m_2 x_2 + \cdots + m_k x_k = 0
    \]
    and \( d \leq m_1 < \operatorname{order}(x_1) \), and so this contradicts the choice of \( \left\lbrace x_1, \cdots, x_k \right\rbrace \).
\end{proof}

The following corollary can be used to show the units of a finite field is a cyclic group.
\begin{corollary}
  \label{corollary-commutative-group-cyclic-if-condition}
  % TODO
  A finite commutative group is cyclic if, for each \( n > 0 \), it contains at most \( n \) elements of order dividing \( n \).
\end{corollary}
\begin{proof}
  By \ref{theorem-finite-generated-group-has-basis}, we can now suppose that \( G = C_{n_1} \times \cdots \times C_{n_r} \) for \( n_i \in \mathbb{Z}_{> 0} \).
  If \( n \) divides \( n_i \) and \( n_j \) with \( i \neq j \), then \( G \) has more than \( n \) elements of order dividing \( n \).
  Therefore, the hypothesis implies that the \( n_i \) are relatively prime.
  Let \( a_i \) generate the \( i \)th factor.
  Then \( (a_1 \cdots a_r ) \) has order \( n_1\cdots n_r \), and so generates \( G \).
\end{proof}

\begin{theorem}
  \label{theorem-decomposition-of-finite-generated-commutative-group}
  A nonzero finitely generated commutative group \( M \) can be expressed
  \[
    M \simeq C_{n_1} \times \cdots \times C_{n_s} \times C^r_\infty
  \]
  for certain integers \( n_1, \cdots, n_s \geq 2 \) and \( r \geq 0 \).
  Moreover,
  \begin{enumerate}
    \item \( r \) is uniquely determined by \( M \).
    \item the \( n_i \) can be chosen so that \( n_1 \geq 2 \) and \( n_1 \mid n_2, \cdots, n_{s - 1} \mid n_s \), and then they are uniquely determined by \( M \).
    \item the \( n_i \) can be chosen to be powers of prime numbers, and then they are uniquely determined by \( M \).
      In other words, \( M \) can be expressed
        \begin{equation}
            M \simeq C_{p^{e_1}_1} \times \cdots \times C_{p^{e_t}_t} \times C^r_\infty,\quad e_i \geq 1, \label{equation-elementary-decomposition}
        \end{equation}
      for certain prime power \( p^{e_i}_i \)(repetitions of primes allowed) uniquely determined by \( M \).
  \end{enumerate}
\end{theorem}

\begin{proof}
  \begin{itemize}
    %TODO
    \item 1. For a prime \( p \) not dividing any of the \( n_i \)
      \[
        M / p M \simeq (C_\infty / p C_\infty)^r \simeq (\mathbb{Z} / p \mathbb{Z})^r,
      \]
      and so \( r \) is the dimension of \( M / p M \) as an \( \mathbb{F}_p \)-vector space.
    \item 2. and 3.
      If \( \gcd(m, n) = 1 \), then \( C_m \times C_n \) contains an element of order \( mn \), and so
      \[
        C_m \times C_n \simeq C_{mn}.
      \]
      Use the above equation to decompose the \( C_{n_i} \) into products of cyclic groups of prime power order.
      Once this has been achieved, it can be used to combine factors to achieve a decomposition as in (2): for example, \( C_{n_s} = \prod C_{p_i}^{e_i} \), where the product is over the distinct primes among the \( p_i \) and \( e_i \) is the highest exponent for the prime \( p_i \).

      In proving the uniqueness statements, we can replace \( M \) with its torsion subgroup(and so assume \( r = 0 \)).
      A prime \( p \) will occur as one of the primes \( p_i \) in \ref{equation-elementary-decomposition} \( \iff M \) has an element of order \( p \), in which case \( p \) will occur exactly \( a \) times, where \( p^a \) is the number of elements of order dividing \( p \).
      Similarly, \( p^2 \) will divide some \( p^{e_i}_i \) in \ref{equation-elementary-decomposition} \( \iff M  \) has an element of order \( p^2 \), in which case it will divide exactly \( b \) of the \( p^{e_i}_i \), where \( p^{a - b}p^{2b} \) is the number of elements in \( M \) of order dividing \( p^2 \).
      Continuing in this fashion, we find that the elementary divisors of \( M \) can be read off from knowing the numbers of elements of \( M \) of each prime power order.

      The uniqueness of the invariant factors can be derived from that of the elementary divisors.
  \end{itemize}
\end{proof}

\begin{definition}
  \label{definition-rank}
  \label{definition-invariant-factor}
  \label{definition-elementary-divisors}
  In \ref{theorem-decomposition-of-finite-generated-commutative-group},
  \begin{itemize}
    \item The number \( r \) is called the \emph{rank} of \( M \).
    \item \( n_1, \cdots, n_s \) are called the \emph{invariant factors} of \( M \).
    \item \( p^{e_1}_{1}, \cdots, p^{e_t}_t \) are called the \emph{elementary divisors} of \( M \).
  \end{itemize}
\end{definition}

\subsection{Dual Groups}
\label{subsection-dual-groups}

Let \( \mu(\mathbb{C}) = \left\lbrace z \in \mathbb{C}: \left\vert z \right\vert = 1 \right\rbrace \).
Then \( \mu(\mathbb{C}) \) is an infinite group.
For any integer \( n \), the set \( \mu_n(\mathbb{C}) \) of elements of orer dividing \( n \) is cycle of order \( n \), i.e.
\[
  \mu_n(\mathbb{C}) := \left\lbrace e^{2\pi i m / n}: 0 \leq m \leq n - 1 \right\rbrace = \left\lbrace 1, \xi, \cdots, \xi^{n - 1} \right\rbrace,
\]
where \( \xi = e^{2 \pi i / n} \) is a primitive \( n \)th root of \( 1 \).

\begin{definition}
  \label{definition-linear-character}
  \label{definition-prinicipal-character}
  \begin{enumerate}
    \item A \emph{linear character}(or just \emph{character}) of a group \( G \) is a homomorphism \( G \to \mu(\mathbb{C}) \).
    \item The homomorphism \( a \mapsto 1 \) is called the \emph{trivial} (or \emph{principal}) \emph{character}.
  \end{enumerate}
\end{definition}

\begin{example}
  \label{example-Legendre-symbol}
  The \emph{Legendre symbol} modulo \( p \) of an integer \( a \) not divisible by \( p \) is
  \[
    \left(\frac{a}{p}\right) := \begin{cases}
      1 &\text{if } a \text{ is a square in } \mathbb{Z} / p\mathbb{Z}\\
      -1 & \text{otherwise}
    \end{cases}.
  \]
  Clearly, this depends only on \( a \) modulo \( p \).
  If neither \( a \) nor \( b \) is divisible by \( p \), then \( \left(\frac{ab}{p}\right) = \left(\frac{a}{p}\right)\left(\frac{b}{p}\right) \): suppose that \( p \neq 2 \);
  then it follows from \( (\mathbb{Z} / p \mathbb{Z})^\times \) is a cyclic group by \ref{example-units-of-finite-field-cyclic} and if \( a, b \in (\mathbb{Z} / p \mathbb{Z})^\times \) are generators, then \( b = a^{n} \) for some odd number \( n \)
  (or else, \( a = a^{m} \) for some even number \( m \), but \( (\mathbb{Z}/ p \mathbb{Z})^\times \) with an even order \( p - 1 \) contradicting with \ref{proposition-generator-of-cyclic-group}).
  Therefore \( [a] \mapsto \left(\frac{a}{p}\right): \left(\mathbb{Z} / p \mathbb{Z}\right)^\times \to \left\lbrace \pm 1 \right\rbrace = \mu_2(\mathbb{C}) \) is a character of \( \left(\mathbb{Z} / p \mathbb{Z}\right)^\times \), sometimes called the quadratic character.
\end{example}

\begin{definition}
  \label{definition-dual-group}
  The set of characters of a group \( G \) becomes a group \( G^\vee \) under the addition,
  \[
    (\chi + \chi')(g) := \chi(g) \chi'(g),
  \]
  called the \emph{dual group} of \( G \).
  For example, the dual group \( \mathbb{Z}^\vee \) of \( \mathbb{Z} \) is isomorphic to \( \mu(\mathbb{C}) \) by the map \( \chi \mapsto \chi(1) \).
\end{definition}

\begin{theorem}
  \label{theorem-duality-of-finite-commutative-group}
  Let \( G \) be a finite commutative group.
  \begin{enumerate}
    \item The dual of \( G^\vee \) is isomorphic to \( G \).
    \item The map \( G \to G^{\vee\vee} \) sending an element \( a \) of \( G \) to the character \( \chi \to \chi(a) \) of \( G^{\vee} \) is an isomorphism.
  \end{enumerate}
  In other words, \( G \simeq G^\vee \) and \( G \simeq G^{\vee\vee} \).
\end{theorem}
\begin{proof}
  The statement is obvious for cyclic groups.
  By \ref{theorem-finite-generated-group-has-basis}, it suffices to see that \( (G \times H)^{\vee} \simeq G^{\vee} \times H^{\vee} \).
  And the homomorphism below is bijective.
  \[
    \chi \mapsto (\chi_1, \chi_2),\quad (G \times H)^{\vee} \to G^\vee \times H^\vee,
  \]
  where \( \chi_1(g) = \chi(g, e) \) and \( \chi_2(h) = \chi(e, h) \).
\end{proof}

\begin{theorem}
  \label{theorem-orthogonality-relation}
  Let \( G \) be a finite commutative group.
  For any characters \( \chi \) and \( \psi \) of \( G \),
  \[
    \sum_{a \in G} \chi(a) \psi(a^{-1}) =
    \begin{cases}
      \left\vert G \right\vert & \text{if } \chi = \psi,\\
      0 &\text{otherwise}
    \end{cases}.
  \]
  In particular,
  \[
    \sum_{a \in G} \chi(a) = \begin{cases}
      \left\vert G \right\vert & \text{ if } \chi \text{ is trivial }\\
      0 & \text{otherwise}
    \end{cases}.
  \]
\end{theorem}
\begin{proof}
  If \( \chi = \psi \), then \( \chi(a) \psi(a^{-1}) = 1 \), and so the sum is \( \left\vert G \right\vert \).
  Otherwise there exists a \( b \in G \) such that \( \chi(b) \neq \psi(b) \).
  As \( a \) runs over \( G \), so also does \( ab \) and so
  \[
    \sum_{a \in G} \chi(a) \psi(a^{-1}) = \sum_{a \in G} \chi(ab) \psi((ab)^{-1}) = \chi(b)\psi(b)^{-1}\sum_{a \in G}\chi(a)\psi(a^{-1}).
  \]
  Because \( \chi(b)\psi(b)^{-1} \neq 1 \), this implies that \( \sum_{a \in G} \chi(a) \psi(a^{-1}) = 0 \).
\end{proof}

\begin{corollary}
  \label{corollary-orthogonality-relation}
  For any \( a \in G \),
  \[
    \sum_{\chi \in G^{\vee}} \chi(a) = \begin{cases}
      \left\vert G \right\vert & \text{ if } a = e\\
      0 &\text{otherwise}
    \end{cases}.
  \]
\end{corollary}

\subsection{Order of elements}
\label{subsection-order-of-elements}

\begin{proposition}
  \label{proposition-order-of-commutative-ab}
  %TODO
  Let \( a, b \) be two elements with finite orders in a group with \( ab = ba \), then \( \left\lvert ab \right\rvert = \gcd(a, b) \).
\end{proposition}

\begin{theorem}
  \label{theorem-general-order-of-ab}
  For any integers \( m, n, r > 1 \), there exists a finite group \( G \) with elements \( a \) and \( b \) such that \( a \) has order \( m \), \( b \) has order \( n \), and \( ab \) has order \( r \).
\end{theorem}

\subsection{Groups of small order}
\label{subsection-groups-of-small-order}

In the following table, \( c + n = t \) means that there are \( c \) commutative groups and \( n \) noncommutative groups.
\begin{table}[H]
  \centering
  \begin{tabular}{c|c|c}
    \hline
    \( \left\vert G \right\vert \) & \( c + n = t \) & Groups\\
    \hline
    \( 4 \) & \( 2 + 0 = 2 \) & \( C_4, C_2 \times C_2 \)\\
    \( 6 \) & \( 1 + 1 = 2 \) & \( C_6; S_3 \)\\
    \hline
  \end{tabular}
\end{table}

\section{Free Groups}
\label{section-free-groups}

\subsection{Free Monoid}
\label{subsection-free-monoid}

\begin{definition}
  \label{definition-word}
  \label{definition-free-monoid}
  Let \( X = \left\lbrace a, b, c, \cdots \right\rbrace \) be a set of symbols.
  A \emph{word} is a finite sequence of symbols from \( X \) in which repetition is allowed.
  For example,
  \[
    aa,\quad aabac,\quad b
  \]
  are distinct words.
  Two words can be ultiplied by \emph{juxtaposition}, for example,
  \[
    aaaa * aabac = aaaaaabac.
  \]
  This defines on the set of all words an associative binary operation.
  The empty sequence is allowed, and we denoted it by \( 1 \).
  Then \( 1 \) serves as an identity element.
  Write \( SX \) for the set of words together with this binary operation.
  Then \( SX \) is a monoid, called the \emph{free monoid} on \( X \).
\end{definition}

\begin{proposition}
  \label{proposition-free-monoid-universal-property}
  When we identify an element \( a \) of \( X \) with the word \( a \), \( X \) becomes a subset of \( SX \) and generates it, i.e. no proper submonoid of \( SX \) contains \( X \).
  Moreover, the map \( X \to SX \) has the following universal property: for any map of sets \( \alpha: X \to S \) from \( X \) to a monoid \( S \), there exists a unique homomorphism \( SX \to S \) making the diagram commute.
\begin{center}
  \begin{tikzcd}
    X & SX \\
    & S
    \arrow["{a \mapsto a}", from=1-1, to=1-2]
    \arrow["\alpha"', from=1-1, to=2-2]
    \arrow[dashed, from=1-2, to=2-2]
  \end{tikzcd}
\end{center}
\end{proposition}

\subsection{Free Group}
\label{subsection-free-group}

We want to construct a group \( FX \) containing \( X \) and having the same universal property as \( SX \) with ``monoid'' replaced by ``group''.
Define \( X' \) to be the set consisting of the symbols in \( X \) and also one addition symbol, denoted \( a^{-1} \), for each \( a \in X \); thus
\[
  X' = \left\lbrace a, a^{-1}, b, b^{-1}, \cdots \right\rbrace.
\]
Let \( W' \) be the set of words using symbols from \( X' \).
This becomes a monoid under juxtaposition, but it is not a group because \( a^{-1} \) is not yet the inverse of \( a \), and we can't cancel out the obvious terms in words of the following form:
\[
  \cdots a a^{-1} \cdots \text{ or } \cdots a^{-1}a \cdots
\]
\begin{definition}
  \label{definition-reduced-word}
  A word is said to be \emph{reduced} if it contains no pairs of the form \( a a^{-1} \) or \( a^{-1} a \).
\end{definition}

Starting with a word \( w \), we can perform a finite sequence of cancellations to arrive at a reduced word(possibly empty), which will be called the \emph{reduced form} \( w_0 \) of \( w \).
There may be many different ways of performing the cancellations, for example
\begin{align*}
  ca\underline{bb^{-1}}a^{-1}c^{-1}ca \to c\underline{aa^{-1}}c^{-1}ca \to \underline{cc^{-1}}ca \to ca\\
  cabb^{-1}a^{-1}\underline{c^{-1}c} a\to ca bb^{-1} \underline{a^{-1}a} \to  ca \underline{bb^{-1}} \to ca.
\end{align*}
We ended up with the same answer, and the next result says that this always happens.
\begin{proposition}
  \label{proposition-unique-reduced-form}
  There is only one reduced form of a word.
\end{proposition}
\begin{proof}
  Use induction on the length of the word \( w \).
  Assume that \( w \) is not reduced and a pair of the form \( a_0 a^{-1}_0 \) occurs.
  Observe that
  \begin{enumerate}
    \item Any two reduced forms of \( w \) obtained by a sequence of cancellations in which \( a_0 a_0^{-1} \) is cancelled first are equal by induction.
    \item Any two reduced forms of \( w \) obtained by a sequence of cancellations in which \( a_0 a^{-1}_0 \) is cancelled at some point are equal, because the result of such a sequence of cancellations will not be affected if \( a_0 a_0^{-1} \) is cancelled first.
    \item Finally, consider a reduced form \( w_0 \) obtained by a sequence in which no cancellation cancels \( a_0 a_0^{-1} \) directly.
      Since \( a_0 a_0^{-1} \) does not remain in \( w_0 \), at least one of \( a_0 \) or \( a_0^{-1} \) must be cancelled at some point.
      If the pair itself is not cancelled, then the first cancellation involving the pair must look like
      \[
        \cdots \not{a}_0^{-1} \overline{\not{a}_0a_0^{-1}} \cdots \text{ or } \cdots \underline{a_0 \not{a}_0^{-1}} \not{a}_0 \cdots.
      \]
      But the word obtained after this cancellation is the same as if our original pair were cancelled, and so we may cancel the original pair instead.
  \end{enumerate}
\end{proof}


\begin{definition}
  \label{definition-equivalent-word}
  We say two words \( w, w' \) are \emph{equivalent}, denoted \( w \sim w' \), if they have the same reduced form.
  This is an equivalence relation.
\end{definition}

\begin{proposition}
  \label{proposition-product-equivalent-words}
  Products of equivalent words are equivalent, i.e.,
  \[
    w \sim w',\quad v \sim v' \implies wv \sim w'v'.
  \]
\end{proposition}

\begin{definition}
  \label{definition-free-group}
  Let \( FX \) be the set of equivalence classes of words.
  Then \( FX \) is a group, called the \emph{free group} on \( X \).
\end{definition}

\begin{proposition}
  \label{proposition-free-group-universal-property}
  For any maps of sets \( \alpha: X \to G \) from \( X \) to a group \( G \), there exists a unique homomorphism \( FX \to G \) making the following diagram commute:
  \begin{center}
    \begin{tikzcd}
      X & FX \\
      & G
      \arrow["{a \mapsto a}", from=1-1, to=1-2]
      \arrow["\alpha"', from=1-1, to=2-2]
      \arrow[dashed, from=1-2, to=2-2]
    \end{tikzcd}
  \end{center}
\end{proposition}

\begin{corollary}
  \label{corollary-any-group-is-a-quotient}
  Every group is a quotient of a free group.
\end{corollary}


\begin{theorem}[Nielsen-Schreier]
  \label{theorem-Nielsen-Schreier}
  Subgroups of free group are free.
\end{theorem}

Two free groups \( FX \) and \( FY \) are isomorphic \( \iff X \) and \( Y \) have the same cardinality.
Thus we can define the \emph{rank} of a free group \( G \) to be the cardinality of any free generating set, where a \emph{free generating set} is a subset \( X \) of \( G \) for which the homomorphism \( FX \to G \) given by \ref{proposition-free-group-universal-property} is an isomorphism.

Let \( H \) be a finitely generated subgroup of a free group \( G \).
Then there is an algorithm for constructing from any finite set of generators for \( H \) a free finite set of generators.
If \( G \) has finite rank \( n \) and \( (G: H) = i < \infty \), then \( H \) is free of rank
\[
  ni - i + 1.
\]
In particular, \( H \) may have rank greater than that of \( G \).

\subsection{Generators and relations}
\label{subsection-generators-and-relations}

Consider a set \( X \) and a set \( R \) of words made up of symbols in \( X \).
\begin{definition}
  \label{definition-generator}
  \label{definition-relation}
  \label{definition-presentation}
  Each element of \( R \) represents an element of the free group \( FX \), and the quotient \( G \) of \( FX \) by the normal subgroup generated by these elements is said to have \( X \) as \emph{generators} and \( R \) as \emph{relations}(or as a \emph{set of defining relations}).
  One also says that \( (X, R) \) is a \emph{presentation} for \( G \), and denoted by \( \left\langle X \mid R \right\rangle \).
\end{definition}

\begin{example}
  \label{example-dihendral-group-presentation}
  The dihendral group \( D_n \) has generators \( r, s \) and defining relations
  \[
    r^n, s^2, srsr.
  \]
\end{example}

\begin{example}
  \label{example-generalized-quaternion-group-presentation}
  The \emph{generalized quaternion group} \( Q_n \), \( n \geq 3 \), has generators \( a, b \) and relations
  \[
    a^{2^{n - 1}} = 1, a^{2^{n - 2}} = b^2, bab^{-1} = a^{-1}.
  \]
  It has order \( 2^n \).
\end{example}

\begin{proposition}
  \label{proposition-presentation-universal-property}
  Let \( G \) be the group defined by the presentation \( (X, R) \).
  For any group \( H \) and map of sets \( \alpha: X \to H \) sending each element of \( R \) to \( 1 \), there exists a unique homomorphism \( G \to H \) making the following diagram commute:
  \begin{center}
    \begin{tikzcd}
      X & G \\
      & H
      \arrow["{a \mapsto a}", from=1-1, to=1-2]
      \arrow["\alpha"', from=1-1, to=2-2]
      \arrow[dashed, from=1-2, to=2-2]
    \end{tikzcd}
  \end{center}
\end{proposition}
\begin{proof}
  By \ref{proposition-free-group-universal-property}, we know that \( \alpha \) can be extended to a homomorphism \( FX \to H \), which we denote again \( \alpha \).
  Then the normal subgroup \( N \) generated by \( \iota R \) is contained in \( \ker \alpha \).
  Finally apply \ref{proposition-kernel-universal-property}.
\end{proof}

\begin{example}
  \label{example-presentation-universal-property}
  Let \( G = \left\lbrace a, b \mid a^n, b^2, baba \right\rbrace \).
  We prove that \( G \) is isomorphic to the dihedral group \( D_n \): the map
  \[
    \left\lbrace a, b \right\rbrace \to D_n,\quad a \mapsto r,\quad b \mapsto s
  \]
  extends uniquely to a homomorphism \( G \to D_n \).
\end{example}

\section{Automorphisms and Extensions}
\label{section-automorphisms-and-extensions}

\subsection{Automorphisms of groups}
\label{subsection-automorphisms-of-groups}


\begin{definition}
  \label{definition-automorphism}
  An \emph{automorphism} of a group \( G \) is an isomorphism of the group with itself.
  The set \( \operatorname{Aut}(G) \) of automorphisms of \( G \) becomes a group under composition.
\end{definition}

\begin{definition}
  \label{definition-inner-automorphism}
  \label{definition-outer-automorphism}
  For \( g \in G \), the map \( i_g \) ``conjugation by \( g \)''
  \[
    x \mapsto g x g^{-1}:\quad G \to G
  \]
  is an automorphism of \( G \).
  An automorphism of this form is called an \emph{inner automorphism}, and the remaining automorphisms are said to be \emph{outer}.
\end{definition}

\begin{proposition}
  \label{proposition-inner-automorphisms-properties}
  \begin{enumerate}
    \item \( G / Z(G) \simeq \operatorname{Inn}(G) \).
    \item \( \operatorname{Inn}(G) \) is a normal subgroup of \( \operatorname{Aut}(G) \)
  \end{enumerate}
\end{proposition}
\begin{proof}
\begin{enumerate}
  \item \( (gh) x (gh)^{-1} = g(h x h^{-1})g^{-1} \), i.e. \( i_{gh}(x) = (i_g \circ i_h)(x) \), and so the map \( g \mapsto i_g: G \to \operatorname{Aut}(G) \) is a homomorphism, with kernel \( Z(G) \).
  \item for \( g \in G \) and \( \alpha \in \operatorname{Aut}(G) \), we have \( \alpha \circ i_g \circ \alpha^{-1} = i_{\alpha(g)} \).
\end{enumerate}
\end{proof}

\begin{example}
  \label{example-linear-space-automorphism}
  Let \( G = (\mathbb{F}^n_p, +) \).
  The automorphisms of \( G \) as a commutative group are just the automorphisms of \( G \) as a vector space over \( \mathbb{F}_p \).
  Thus \( \operatorname{Aut}(G) = \operatorname{GL}_n(\mathbb{F}_p) \).
  Because \( G \) is commutative, all nontrivial automorphisms of \( G \) are outer.
  In particular, \( \operatorname{Aut}(C_2 \times C_2) \simeq \operatorname{GL}_2(\mathbb{F}_2) \). 
\end{example}

\begin{example}
  \label{example-quaternion-group-automophism}
  As the centre of the quaternion group \( Q \) is \( \left\langle a^2 \right\rangle \),
  \[
    \operatorname{Inn}(Q) \simeq Q / \left\langle a^2 \right\rangle \simeq C_2 \times C_2.
  \]
  In fact, \( \operatorname{Aut}(Q) \simeq S_4 \).
\end{example}

\subsection{Automorphisms of Cyclic Groups}
\label{subsection-automorphisms-of-cyclic-groups}

Let \( G \) be a finite cyclic group with order \( n \).
An automorphism \( \alpha \) of \( G \) must send \( \alpha \) to another generator of \( G \), and so \( \alpha(a) = a^m \) for some \( m \) relatively prime to \( n \), by \ref{proposition-generator-of-cyclic-group}.
The map \( \alpha \mapsto m \) defines an isomorphism
\[
  \operatorname{Aut}(C_n) \to (\mathbb{Z} / n \mathbb{Z})^\times,
\]
where
\[
  (\mathbb{Z} / n \mathbb{Z})^\times = \left\lbrace \text{units in the ring } \mathbb{Z} / n \mathbb{Z} \right\rbrace = \left\lbrace m + n \mathbb{Z}: \gcd(m, n) = 1 \right\rbrace.
\]
It remains to determine \( (\mathbb{Z} / n \mathbb{Z})^\times \).
If \( n = p^{r_1}_1 \cdots p^{r_s}_s \) is the factorization of \( n \) into a product of powers distinct primes, then
\[
  \mathbb{Z} / n \mathbb{Z} \simeq \mathbb{Z} / p^{r_1}\mathbb{Z} \times \cdots \times \mathbb{Z} / p^{r_s}_s \mathbb{Z},\quad m \mod n \leftrightarrow (m \mod p_1^{r_1}, \cdots, n \mod p_s^{r_s})
\]
by Chinese Remainder Theorem.
And so
\[
  (\mathbb{Z} / n \mathbb{Z})^\times \simeq (\mathbb{Z} / p^{r_1}_1 \mathbb{Z})^\times \times \cdots \times (\mathbb{Z} / p^{r_s}_s \mathbb{Z})^\times.
\]
It remains to consider the case \( n = p^r \), where \( p \) is prime.
\( (\mathbb{Z} / p^r \mathbb{Z})^\times \) has order \( p^{r - 1}(p - 1) \).
The homomorphism
\[
  (\mathbb{Z} / p^r \mathbb{Z})^\times \to (\mathbb{Z} / p \mathbb{Z})^\times
\]
is surjective with kernel of order \( p^{r - 1} \), and we know that \( (\mathbb{Z} / p \mathbb{Z})^\times \) is cyclic by \ref{example-units-of-finite-field-cyclic}.
Let \( a \in (\mathbb{Z} / p^r \mathbb{Z})^\times \) map to a generators of \( (\mathbb{Z} / p \mathbb{Z})^\times \).
Then \( a^{p^r(p - 1)} = \left(a^{p^{r - 1}(p - 1)}\right)^{p} = 1 \).
And \( a^{p^r} \neq 1 \) in \( (\mathbb{Z} / p \mathbb{Z})^\times \), hence \( a^{p^r} \) again maps to a generator of \( (\mathbb{Z} / p \mathbb{Z})^\times \).
Therefore, \( (\mathbb{Z} / p^r \mathbb{Z})^\times \) contains an element \( \xi := a^{p^r} \) of order \( p - 1 \).
By \ref{lemma-automorphisms-of-cyclic-group-1}, we have  that \( 1 + p \) has order \( p^{r - 1} \) in \( (\mathbb{Z} / p^r \mathbb{Z})^\times \).
Therefore \( (\mathbb{Z} / p^r \mathbb{Z})^\times \) is cyclic with generator \( \xi \cdot (1 + p) \) and every element can be written uniquely in the form
\[
  \xi^i \cdot (1 + p)^j,\quad 0 \leq i < p - 1,\quad 0 \leq j < p^{r - 1}.
\]
On the other hand,
\[
  (\mathbb{Z} / 8\mathbb{Z})^{\times} = \left\lbrace \bar{1}, \bar{3}, \bar{5}, \bar{7} \right\rbrace = \left\langle \bar{3}, \bar{5} \right\rangle \simeq C_2 \times C_2
\]
is not cyclic.

In summary, we have(For \( p = 2 \), see \ref{lemma-automorphisms-of-cyclic-group-2})
\begin{theorem}
  \label{theorem-automorphisms-of-cyclic-group}
  \begin{enumerate}
    \item For a cyclic group of \( G \) of order \( n \), \( \operatorname{Aut}(G) \simeq (\mathbb{Z} / n \mathbb{Z})^\times \).
      The automorphism of \( G \) of \( G \) corresponding to \( [m] \in (\mathbb{Z} / n\mathbb{Z})^{\times} \) is \( a \mapsto a^m \).
    \item If \( n = p^{r_1}_1 \cdots p^{r_s}_s \) with the \( p_i \) distinct primes, then
      \[
        (\mathbb{Z} / n\mathbb{Z})^\times \simeq (\mathbb{Z} / p^{r_1}_1\mathbb{Z})^\times \times \cdots \times (\mathbb{Z} / p_s^{r_r}\mathbb{Z})^\times,\quad m \mod{n} \leftrightarrow (m \mod{p_1^{r_1}}, \cdots, m \mod{p_s^{r_s}}).
      \]
    \item For a prime \( p \),
      \[
        (\mathbb{Z} / p^r \mathbb{Z})^\times \simeq \begin{cases}
          C_{(p - 1)p^{r - 1}} & p \text{ odd }\\
          C_2 & p^r = 2^2\\
          C_2 \times C_{2^{r - 2}} & p = 2, r > 2.
        \end{cases}
      \]
  \end{enumerate}
\end{theorem}
\begin{lemma}
  \label{lemma-automorphisms-of-cyclic-group-1}
  \begin{enumerate}
    \item Let \( n \) and \( k \) be integers, with \( n \geq 2 \) and \( k \geq 0 \).
      Then
      \[
        (1 + n)^{n^k} \equiv 1 \pmod{n^{k + 1}}.
      \]
    \item If \( p \) is an odd prime, then
      \[
        (1 + p)^{p^k} \equiv 1 + p^{k + 1} \pmod{p^{k + 2}}
      \]
      for every positive integer \( k \).
    \item If \( p \) is an odd prime, then
      \[
        (1 + p)^{p^k} \not\equiv 1 \pmod{p^{k + 2}}
      \]
      for all \( k \geq 0 \).
    \item Let \( p \) be an odd prime, and \( n \) positive integer.
      Then the order of \( \overline{1 + p} \in (\mathbb{Z} / p^{n}\mathbb{Z})^{\times} \) is \( p^{n - 1} \).
  \end{enumerate}
\end{lemma}
\begin{proof}
  \href{https://math.stackexchange.com/questions/238414/showing-1p-is-an-element-of-order-pn-1-in-mathbbz-pn-mathbbz-t}{stackexchange}
\end{proof}

\begin{lemma}
  \label{lemma-automorphisms-of-cyclic-group-2}
  \begin{enumerate}
    \item \( (1 + 4)^{2^{n - 3}} \in (\mathbb{Z} / 2^n \mathbb{Z})^\times \) and the element \( 5 \) has order \( 2^{n - 2} \) for \( n \geq 2 \).
    \item \( 5 \) and \( -1 \) generate the group \( (\mathbb{Z} / 2^n \mathbb{Z})^{\times} \).
    \item \( -1 \notin \left\langle 5 \right\rangle \).
    \item \( (\mathbb{Z} / 2^n\mathbb{Z})^{\times} \simeq \mathbb{Z} / 2 \mathbb{Z} \times \mathbb{Z} / 2^{n - 2} \mathbb{Z}  \).
  \end{enumerate}
\end{lemma}
\begin{proof}
  \href{https://math.stackexchange.com/questions/459815/the-structure-of-the-group-mathbbz-2n-mathbbz}{stackexchange}
\end{proof}

\begin{definition}
  \label{definition-characteristic-subgroup}
  A \emph{characteristic subgroup} of a group \( G \) is a subgroup \( H \) such that \( \alpha(H) = H \) for all automorphisms \( \alpha \) of \( G \).
\end{definition}

\begin{remark}
  \label{remark-verify-subgroup-characteristic}
  Like \ref{remark-verify-subgroup-normal}, to show \( H \) is a subgroup is to check that \( \alpha(H) \subseteq H \) for all \( \alpha \in \operatorname{Aut}(G) \).
  Moreover, a subgroup \( H \) of \( G \) is normal if it is stable under all \emph{inner automorphisms} of \( G \), and it is characteristic if it is stable under all automorphisms.
  In particular, a characteristic subgroup is normal.
\end{remark}

\begin{remark}
  \label{remark-characteristic-subgroup}
  Consider a group \( G \) and a normal subgroup \( N \). %TODO
  An inner automorphism of \( G \) restricts to an automorphism of \( N \), which may be outer.
  Thus a normal subgroup of \( N \) need not be a normal subgroup of \( G \).
  However, a characteristic subgroup of \( N \) will be a normal subgroup of \( G \).
  Also a characteristic subgroup of a characteristic subgroup is a characteristic subgroup.
\end{remark}

\begin{example}
  \label{example-centre-is-characteristic}
  The centre \( Z(G) \) of \( G \) is a characteristic subgroup.
\end{example}

\begin{example}
  \label{example-only-nontrivial-is-characteristic}
  If \( H \) is the only nontrivial subgroup of \( G \), then it must be characteristic.
\end{example}

Every subgroup of a commutative group is normal but not necessarily characteristic.
\begin{example}
  \label{example-normal-but-not-characteristic}
  Every subspace of dimension \( 1 \) in \( \mathbb{F}^2_p \) is subgroup of \( \mathbb{F}_p^2 \), but it is not characteristic because it is not stable under \( \operatorname{Aut}(\mathbb{F}^2_p) = \operatorname{GL}_2(\mathbb{F}_p) \).
\end{example}

\subsection{Semidirect Products}
\label{subsection-semidirect-products}

Let \( N \) be a normal subgroup of \( G \).
Each element \( g \) of \( G \) defines an automorphism of \( N \), \( n \mapsto g n g^{-1} \), and this defines a homomorphism
\[
  \theta: G \to \operatorname{Aut}(N),\quad g \mapsto \left. i_g \right\vert_N.
\]
If there exists a subgroup \( Q \) of \( G \) such that \( G \to G / N \) maps \( Q \) isomorphiscally onto \( G / N \), then we can reconstruct \( G \) from \( N, Q \), and the restriction of \( \theta \) to \( Q \).
Indeed, an element \( g \) of \( G \) can be written uniquely in the form
\[
  g = nq,\quad n \in N,\quad q \in Q,
\]
since any element \( g \in G \) falls in a unique coset of \( N \).
\( q \) must be the unique element of \( Q \) mapping to \( gN \in G / N \), and \( n \) must be \( gq^{-1} \).
Thus, we have a one-to-one correspondence of sets
\[
  G \mathop{\longleftrightarrow}^{1:1} N \times Q.
\]
If \( g = nq \) and \( g' = n'q' \), then
\[
  gg' = (nq)(n'q') = n(qn'q^{-1})qq' = n \cdot \theta(q)(n') \cdot qq'.
\]
\begin{definition}
  \label{definition-semidirect-product}
  A group \( G \) is a \emph{semidirect product} of its subgroups \( N \) and \( Q \) if \( N \) is normal and the homomorphism \( G \to G / N \) induces an isomorphism \( Q \to G / N \).
  Equivalently, \( G \) is a semidirect product of subgroup \( N \) and \( Q \) if
  \[
    N \triangleleft G;\quad NQ = G;\quad N \cap Q = \left\lbrace 1 \right\rbrace.
  \]
  When \( G \) is the semidirect product of subgroups \( N \) and \( Q \), we write \( G = N \rtimes Q \)(or \( N \rtimes_\theta Q \), where \( \theta: Q \to \operatorname{Aut}(N) \)).
\end{definition}

\begin{example}
  \label{example-dihendral-group-as-semidirect-product}
  In \( D_n \), \( n \geq 2 \), let \( C_n = \left\langle r \right\rangle \) and \( C_2 = \left\langle s \right\rangle \); then
  \[
    D_n = \left\langle r \right\rangle \rtimes_{\theta} \left\langle s \right\rangle = C_n \rtimes C_2,
  \]
  where \( \theta(s)(r^i) = r^{-i} \).
\end{example}

\begin{example}
  \label{example-symmetric-group-as-semidirect-product}
  The alternating subgroup \( A_n \) is a normal subgroup of \( S_n \) and \( C_2 = \left\langle (12) \right\rangle \) maps isomorphically onto \( S_n / A_n \).
  Therefore \( S_n = A_n \rtimes C_2 \).
\end{example}

\begin{example}
  \label{example-upper-triangular-matrices-as-semidirect-product}
  Let \( G = \operatorname{GL}_n(F) \).
  Let \( B \) be the subgroup of upper triangular matrices in \( G, T \) the subgroup of diagonal matrices in \( G \), and \( U \) the subgroup of upper triangular matrices with all their diagonal coefficients equal to \( 1 \).
  When \( n =2 \), \( U \) is a normal subgroup of \( B \), \( UT = B \), and \( U \cap T = \left\lbrace 1 \right\rbrace \).
  Therefore,
  \[
    B = U \rtimes T.
  \]
  Note that, when \( n \geq 2 \), the action of \( T \) on \( U \) is not trivial, for example
  \[
    \begin{pmatrix}
      a & 0\\ 0 & b
    \end{pmatrix} \begin{pmatrix}
      1 & 0\\ 0 & 1
    \end{pmatrix}\begin{pmatrix}
      a^{-1} & 0\\ 0 & b^{-1}
    \end{pmatrix} = \begin{pmatrix}
      1 & ac/b\\ 0 & 1
    \end{pmatrix},
  \]
  and so \( B \) is not the direct product of \( T \) and \( U \).
\end{example}

\begin{example}
  \label{example-group-without-nontrivial-semidirect-product}
  \begin{enumerate}
    \item The quaternion group can not be written as a semidirect product in any nontrivial fashion.
    \item A cyclic group of order \( p^2 \), \( p \) prime, is not a semidirect product, because it has only one subgroup of order \( p \).
  \end{enumerate}
\end{example}

We have seen that, from a semidirect product \( G = N \rtimes Q \), we obtain a triple
\[
  (N, Q, \theta: Q \to \operatorname{Aut}(N)),
\]
and that the triple determines \( G \).

\begin{proposition}
  \label{proposition-semidirect-composite-group}
  Every triple \( (N, Q, \theta) \) consisting of two groups \( N \) and \( Q \) and a homomorphism \( \theta: Q \to \operatorname{Aut}(N) \).
As a set, let \( G = N \times Q \), and define
\[
  (n, q) (n', q') = (n \cdot \theta(q)(n'), qq').
\]
Then \( G \) is a group, and, in fact, the semidirect product of \( N \) and \( Q \).
\end{proposition}
\begin{proof}
  Write \( {}^qn \) for \( \theta(q)(n) \), so that the composition law becomes
  \[
    (n, q)(n', q') = (n\cdot {}^qn', qq').
  \]
  Then
  \[
    ((n, q), (n', q'))(n'', q'') = (n \cdot {}^q n' \cdot {}^{qq'}n'', qq'q'') = (n, q)((n', q')(n'', q''))
  \]
  and so the asscociative law holds.
  Because \( \theta(1) = 1 \) and \( \theta(q)(1) = 1 \),
  \[
    (1, 1) (n, q) = (n, q) = (n, q)(1, 1),
  \]
  and so \( (1, 1) \) is an identity element.
  Next
  \[
    (n, q)({}^{q^{-1}}n^{-1}, q^{-1}) = (1, 1) ({}^{q^{-1}}n^{-1}, q^{-1})(n, q),
  \]
  and so \( ({}^{q^{-1}}n^{-1}, q^{-1}) \) is an inverse for \( (n, q) \).
  Thus \( G \) is a group, and it is obvious that \( N \triangleleft G, N Q = G \) and \( N \cap Q = \left\lbrace 1 \right\rbrace \) and so \( G = N \rtimes Q \).
\end{proof}


\begin{example}
  \label{example-group-of-order-12}
  A \emph{group of order} \( 12 \).
  Let \( \theta \) be the nontrivial homomorphism
  \[
    C_4 \to \operatorname{Aut}(C_3) \simeq C_2,
  \]
  namely, that sending a generator of \( C_4 \) to the map \( a \mapsto a^2 \).
  Then \( G := C_3 \rtimes_\theta C_4 \) is a noncommutative group of order \( 12 \), not isomorphic to \( A_4 \).
  If we denote the generators of \( C_3 \) and \( C_4 \) by \( a \) and \( b \), then \( a \) and \( b \) generatate \( G \), and have the defining relations
  \[
    a^3 = 1,\quad b^4 = 1,\quad bab^{-1} = a^2.
  \]
\end{example}

\begin{example}
  \label{example-make-outer-automorphisms-inner}
  \emph{Making outer automorphisms inner}.
  Let \( \alpha \) be an automorphism, possibly outer, of a group \( N \).
  We can realize \( N \) as a normal subgroup of a group \( G \) in such a way that \( \alpha \) becomes the restriction to \( N \) of an inner automorphism of \( G \).
  To see this, let \( \theta: C_\infty \to \operatorname{Aut}(N) \) be the homomorphism sending a generator \( a \) of \( C_\infty \) to \( \alpha \in \operatorname{Aut}(N) \), and let \( G = N \rtimes_\theta C_\infty \).
  The element \( g = (1, a) \) of \( G \) has the property that \( g(n, 1)g^{-1} = (\alpha(n), 1) \) for all \( n \in N \).
\end{example}


It will be useful to have criteria for when two triples \( (N, Q, \theta) \) and \( (N, Q, \theta') \) determine isomorphic groups.

\begin{lemma}
  \label{lemma-criteria-semidirect-product-1}
  If there exists an \( \alpha \in \operatorname{Aut}(N) \) such that
  \[
    \theta'(q) = \alpha \circ \theta(q) \circ \alpha^{-1},\quad \text{ all } q \in Q,
  \]
  then the map
  \[
    \gamma: (n, q) \mapsto (\alpha(n), q)\quad N \rtimes_\theta Q \to N \rtimes_{\theta'} Q
  \]
  is an isomorphism.
\end{lemma}
\begin{proof}
  For \( (n, q) \in N \rtimes_\theta Q \), then
  \begin{align*}
    \gamma(n, q) \cdot \gamma(n', q') &= (\alpha(n), q) \cdot (\alpha(n'), q')\\
                                      &=(\alpha(n)\cdot(\alpha \circ \theta(q) \circ \alpha^{-1})(\alpha(n')), qq')\\
                                      &= (\alpha(n) \cdot \alpha(\theta(q)(n')), qq'),
  \end{align*}
  and
  \begin{align*}
    \gamma((n, q) \cdot (n', q')) &= \gamma(n \cdot \theta(q)(n'), qq')\\
                                  &= (\alpha(n)\cdot \alpha(n) \cdot\alpha(\theta(q)(n')), qq').
  \end{align*}
  Therefore \( \gamma \) is a homomorphism.
  The map
  \[
    (n, q) \mapsto (\alpha^{-1}(n, q)): \quad N \rtimes_{\theta'}Q \to N \rtimes_\theta Q
  \]
  is also a homomorphism, and it is inverse to \( \gamma \).
\end{proof}

\begin{lemma}
  \label{lemma-criteria-semidirect-product-2}
  If \( \theta = \theta' \circ \alpha \) with \( \alpha \in \operatorname{Aut}(Q) \), then the map
  \[
    \gamma: (n, q) \mapsto (n, \alpha(q))\quad N \rtimes_\theta Q \simeq N \rtimes_{\theta'}Q
  \]
  is an isomorphism.
\end{lemma}
\begin{proof}
  \begin{align*}
    \gamma(n, q) \cdot \gamma(n', q') &= (n, \alpha(q))(n', \alpha(q'))\\
                                      &= (n \cdot \theta'\circ\alpha(q)(n'), \alpha(qq'))\\
                                      &= (n \cdot \theta(q)n, \alpha(qq'))\\
                                      &= \gamma(n \cdot \theta(q)(n'), qq') = \gamma((n, q)\cdot(n', q')).
  \end{align*}
\end{proof}

\begin{lemma}
  \label{lemma-criteria-semidirect-product-3}
  If \( Q \) is finite and cyclic and the subgroup \( \theta(G) \) of \( \operatorname{Aut}(N) \) is conjugate to \( \theta'(Q) \), then
  \[
    N \rtimes_\theta Q \simeq N \rtimes_{\theta'}  Q.
  \]
\end{lemma}
\begin{proof}
  Let \( a \) generate \( Q \).
  By assumption, there exists an \( \alpha' \in Q \) and an \( \alpha \in \operatorname{Aut}(N) \) such that
  \[
    \theta'(a') = \alpha \cdot \theta(a) \cdot \alpha^{-1}.
  \]
  The element \( \theta'(a') \) generatates \( \theta'(Q) \), and we can choose \( a' \) to generate \( Q \), say \( a' = a^i \).
  Now the map \( (n, q) \mapsto (\alpha(n), q^i) \) is an isomorphism \( N \rtimes_\theta Q \to N \rtimes_{\theta'} Q \), with the inverse \( (n, q^i) \mapsto (\alpha^{-1}(n), q) \).
\end{proof}

\begin{theorem}
  \label{theorem-criteria-semidirect-product}
  Let \( G \) be a group with subgroups \( H_1 \) and \( H_2 \) such that \( G = H_1 H_2 \) and \( H_1 \cap H_2 = \left\lbrace e \right\rbrace \), so that each element \( g \) of \( G \) can be writtern uniquely as \( g = h_1 h_2 \) with \( h_1 \in H_1 \) and \( h_2 \in H_2 \).
  \begin{enumerate}
    \item If \( H_1 \) and \( H_2 \) are both normal, then \( G \) is the direct product of \( H_1 \) and \( H_2 \), \( G = H_1 \times H_2 \).
    \item If \( H_1 \) is normal in \( G \), then \( G \) is the semidirect product of \( H_1 \) and \( H_2 \), \( G = H_1 \rtimes H_2 \).
    \item If neither \( H_1 \) nor \( H_2 \) is normal, then \( G \) is the Zappa-Sz\'{e}p (or knit) product of \( H_1 \) and \( H_2 \).
  \end{enumerate}
\end{theorem}

\subsection{Extensions of Groups}
\label{subsection-extensions-of-groups}


\begin{definition}
  \label{definition-complete-group}
  A group \( G \) is \emph{complete} if the map \( g \mapsto i_g: G \to \operatorname{Aut}(G) \) is an isomorphism.
\end{definition}

\begin{proposition}
  \label{proposition-complete-characteristic}
  A group \( G \) is a complete if and only if
  \begin{enumerate}
    \item the centre \( Z(G) \) of \( G \) is trivial.
    \item every automorphism of \( G \) is inner.
  \end{enumerate}
\end{proposition}

\begin{example}
  \label{example-complete-symmetric-group}
  The group \( S_n \) is complete for \( n \neq 2, 6 \), but \( S_2 \) fails (1) and \( S_6 \) fails (2) in the preceding proposition.
\end{example}

\begin{example}
  \label{example-simple-noncommutative-group-complete}
  If \( G \) is a simple noncommutative group, then \( \operatorname{Aut}(G) \) is complete.
\end{example}


\begin{definition}
  \label{definition-exact-sequence}
  A sequence of groups and homomorphisms
  \begin{equation}
    1 \to N \xrightarrow{\iota} G \xrightarrow{\pi} 1
    \label{equation-exact-sequence}
  \end{equation}
  is \emph{exact} if \( \iota \) is injective, \( \pi \) is surjective, and \( \ker \pi = \operatorname{Im}(\iota) \).
\end{definition}
Thus \( \iota(N) \) is a normal subgroup of \( G \) and \( G / \iota(N) \simeq Q \).
We often identify \( N \) with the subgroup \( \iota(N) \) of \( G \) and \( Q \) with the quotient \( G / N \).

\begin{definition}
  \label{definition-extension}
  \label{definition-central}
  An exact sequence \ref{equation-exact-sequence} is also called an \emph{extension of} \( Q \) by \( N \).
  An extension is \emph{central} if \( \iota(N) \subseteq Z(G) \).
\end{definition}
For example, a semidirect product \( N \rtimes_\theta Q \) gives rise to an extension of \( Q \) by \( N \),
\[
  1 \to N \to N \rtimes_\theta Q \to Q \to 1,
\]
which is central \( \iff \theta \)  is the trivial homomorphism and \( N \) is commutative:
\[
  (n, q)(n', 1)({}^{q^{-1}}n^{-1}, q^{-1}) = (n, q)(n' \cdot {}^{q^{-1}}n^{-1}, q^{-1}) = (n \cdot {}^q (n' \cdot {}^{q^{-1}}n^{-1}), 1)
\]

\begin{definition}
  \label{definition-isomorphic-extension}
  Two extensions of \( Q \) by \( N \) are said to be \emph{isomorphic} if there exists a commutative diagram
  \begin{center}
    \begin{tikzcd}
      1 & N & G & Q & 1 \\
      1 & N & {G'} & Q & 1
      \arrow[from=1-1, to=1-2]
      \arrow[from=1-2, to=1-3]
      \arrow[Rightarrow, no head, from=1-2, to=2-2]
      \arrow[from=1-3, to=1-4]
      \arrow["\simeq"{description}, from=1-3, to=2-3]
      \arrow[from=1-4, to=1-5]
      \arrow[Rightarrow, no head, from=1-4, to=2-4]
      \arrow[from=2-1, to=2-2]
      \arrow[from=2-2, to=2-3]
      \arrow[from=2-3, to=2-4]
      \arrow[from=2-4, to=2-5]
    \end{tikzcd}
  \end{center}
\end{definition}
\begin{definition}
  \label{definition-split-sequence}
  An extension of \( Q \) by \( N \),
  \[
    1 \to N \xrightarrow{\iota} G \xrightarrow{\pi} Q \to 1,
  \]
  is said to be \emph{split} if it is isomorphic to the extension definded by a semidirect product \( N \rtimes_\theta Q \).
\end{definition}
\begin{remark}
  \label{remark-split-equivalent-condition}
  Equivalent conditions:
  \begin{enumerate}
    \item there exists a subgroup \( Q' \subseteq G \) such that \( \pi \) induces an isomorphism \( Q' \to Q \);
    \item there exists a homomorphism: \( s: Q \to G \) such that \( \pi \circ s = \operatorname{id} \).(\( G = \iota(N) \rtimes s(Q) \)).
  \end{enumerate}
\end{remark}

In general, an extension will not split.
\begin{example}
  \label{example-not-split}
  For example,
  \[
    1 \to C_p \to C_{p^2} \to C_p \to 1
  \]
  doesn't split.
  If \( Q \) is the quaternion group and \( N \) is its centre, then
  \[
    1 \to N \to Q \to Q / N \to 1
  \]
  doesn't split.(if it did, \( Q \) would be commutative because \( N \) and \( Q / N \) are commutative and \( \theta \) trivial(\( N \) is its centre))
\end{example}

\begin{theorem}[Schur-Zassenhaus]
  \label{theorem-schur-zessenhaus}
  An extension of finite groups of relatively prime order is split.
\end{theorem}

\begin{proposition}
  \label{proposition-complete-implies-split}
  An extension \ref{equation-exact-sequence} splits if \( N \) is complete.
  In fact, \( G \) is then the direct product of \( N \) with the \emph{centralizer} of \( N \) in \( G \),
  \[
    C_G(N) := \left\lbrace g \in G: gn = ng \text{ all } n \in N \right\rbrace.
  \]
\end{proposition}
\begin{proof}
  Let \( H = C_G(N) \).
  \begin{enumerate}
    \item for any \( g \in G, n \mapsto g n g^{-1}: N \to N \) is an automorphism of \( N \)(\( N \) is already a normal subgroup of \( G \)), and it must be the inner automorphism defined by an element \( \gamma \) of \( N \);
      thus
      \[
        g n g^{-1} = \gamma n \gamma^{-1}\quad \text{ all } n \in N.
      \]
      It implies that \( \gamma^{-1}g \in H \), and hence \( g = \gamma(\gamma^{-1}g) \in N H \).
      Since \( g \) is arbitrary, \( G = NH \).
    \item \( N \) is complete and hence \( Z(N) = \left\lbrace e \right\rbrace \).
    \item Finally, for any element \( g = nh \in G \),
      \[
        g H g^{-1} = n(h H h^{-1})n^{-1} = nH n^{-1} = H
      \]
      Therefore, \( H \) is normal in \( G \).
      Hence, \( G = N \times H \) by \ref{proposition-iff-conditions-of-direct-product}.
  \end{enumerate}
\end{proof}

\section{Groups Acting on Sets}
\label{section-groups-acting-on-sets}

\subsection{Actions}
\label{subsection-actions}

\begin{definition}
  \label{definition-action}
  Let \( X \) be a set and let \( G \) be a group.
  A \emph{left action} of \( G \) on \( X \) is a mapping \( (g, x) \mapsto gx: G \times X \to X \) such that
  \begin{enumerate}
    \item \( 1x = x \), for all \( x \in X \);
    \item \( (g_1 g_2) x = g_1 (g_2x) \), all \( g_1, g_2 \in G, x \in X \).
  \end{enumerate}
  A set together with a left action of \( G \) is called a (left) \emph{\( G \)-set}.
  An action is \emph{trivial} if \( gx = x \) for all \( g \in G \).
\end{definition}

The conditions imply that, for each \( g \in G \), left translation by \( g \),
\[
  g_L: X \to X,\quad x \mapsto gx,
\]
has \( (g^{-1})_L \) as an inverse, and therefore \( g_L \) is a bijection, i.e. \( g_L \in \operatorname{Sym}(X) \).
(2) now says that
\[
  g \mapsto g_L: G \to \operatorname{Sym}(X)
\]
is a homomorphism.

\begin{definition}
  \label{definition-faithful}
  The action is said to be \emph{faithful}(or \emph{effective}) if the homomorphism is injective, i.e., if
  \[
    gx = x \text{ for all } x \in X \implies g = 1.
  \]
\end{definition}

\begin{example}
  \label{example-symmetric-group-faithfully-action}
  Every subgroup of the symmetric group \( S_n \) acts faithfully on \( \left\lbrace 1, 2, \cdots, n \right\rbrace \).
\end{example}

\begin{example}
  \label{example-subgroup-action}
  Every subgroup \( H \) of a group \( G \) acts faithfully on \( G \) by left translation,
  \[
    H \times G \to G, \quad (h, x) \mapsto hx.
  \]
\end{example}

\begin{example}
  \label{example-rigid-motion-action}
  The \emph{group of rigid motions} of \( \mathbb{R}^n \) is the group ob bijections \( \mathbb{R}^n \to \mathbb{R}^n \) preserving lengths.
\end{example}

\begin{definition}
  \label{definition-G-map}
  \label{definition-G-map-isomorphism}
  A \( G \)-\emph{map of} \( G \)-sets is a map \( \varphi: X \to Y \) such that
  \[
    \varphi(gx) = g\varphi(x),\quad \text{ all } g \in X,\quad x \in X.
  \]
  An \emph{isomorphism} of \( G \)-sets is a bijective \( G \)-maps; its inverse is then also a \( G \)-map.
\end{definition}


\begin{definition}
  \label{definition-stable-under-action}
  \label{definition-G-orbit}
  Let \( G \) act on \( X \).
  A subset \( S \subseteq X \) is said to be \emph{stable} under the action of \( G \) if
  \[
    g \in G, x \in S \implies gx \in S.
  \]
  The action of \( G \) on \( X \) then induces an action of \( G \) on \( S \).
  Write \( x \sim_G y \) if \( y = gx \), for some \( g \in G \).
  This is an equivalence relation.
  The equivalence classes are called \( G \)-\emph{orbits}.
  Thus the \( G \)-orbits partition \( X \).
  Write \( G \backslash X \) for the set of orbits.
\end{definition}

\begin{remark}
  \label{remark-G-orbit}
  By definition, the \( G \)-orbit containing \( x_0 \) is
  \[
    G x_0 = \left\lbrace g x_0: g \in G \right\rbrace.
  \]
  It is the smallest \( G \)-stable subset of \( X \) containing \( x_0 \).
  And a subset of \( X \) is stable \( \iff \) it is a union of orbits.
\end{remark}

\begin{example}
  \label{example-cyclic-action}
  Suppose \( G \) acts on \( X \), and let \( \alpha \in G \) be an element of order \( n \).
  Then the orbits of \( \left\langle \alpha \right\rangle \) are the sets of the form
  \[
    \left\lbrace x_0, \alpha x_0, \cdots, \alpha^{n - 1}x_0 \right\rbrace.
  \]
  And these elements need not be distinct.
\end{example}


\begin{definition}
  \label{definition-transitive}
  \label{definition-homogeneous}
  The action of \( G \) on \( X \) is said to be \emph{transitive}, and \( G \) is said to act \emph{trasitively} on \( X \), if there is only one orbit.
  The set \( X \) is then called a \emph{homogeneous} \( G \)-set.
\end{definition}

\begin{example}
  \label{example-symmetric-group-transitive-action}
  \( S_n \) acts transitively on \( \left\lbrace 1, 2, \cdots, n \right\rbrace \).
\end{example}

\begin{example}
  \label{example-transitive}
  For any subgroup \( H \) of a group \( G \), \( G \) acts transitively on \( G / H \).
  But the action of \( G \) on itself by conjugation is never transitive if \( G \neq 1 \), because \( \left\lbrace 1 \right\rbrace \) is always a conjugacy class.
\end{example}

\begin{definition}
  \label{definition-doubly-transitive}
  The action of \( G \) on \( X \) is \emph{doubly transitive} if for any two pairs \( (x_1, x_2), (y_1, y_2) \) of elements of \( X \) with \( x_1 \neq x_2 \) and \( y_1 \neq y_2 \), there exists a single \( g \in G \) such that \( gx_1 = y_1 \) and \( gx_2 = y_2 \).
  Define \( k \)-\emph{fold transitivity} for \( k \geq 3 \) similarly.
\end{definition}


\begin{definition}
  \label{definition-stabilizer}
  \label{definition-isotropy-group}
  \label{definition-free-action}
  Let \( G \) act on \( X \).
  The \emph{stabilizer} (or \emph{isotropy group}) of an element \( x \in X \) is
  \[
    \operatorname{Stab}(x) = \left\lbrace g \in G: gx = x \right\rbrace.
  \]
  The action is \emph{free} if \( \operatorname{Stab}(x) = \left\lbrace e \right\rbrace \) for all \( x \).
\end{definition}

\( \operatorname{Stab}(x) \) is a subgroup, but it need not be a normal subgroup, and more precisely, we have the following
\begin{lemma}
  \label{lemma-stabilizer-conjugate}
  For any \( g \in G \) and \( x \in X \),
  \[
    \operatorname{Stab}(gx) = g \cdot \operatorname{Stab}(x) \cdot g^{-1}.
  \]
\end{lemma}

\begin{definition}
  \label{definition-centralizer}
  Let \( G \) act on itself by conjugation.
  Then
  \[
    \operatorname{Stab}(x) = \left\lbrace g \in G: gx = xg \right\rbrace.
  \]
  This group is called the \emph{centralizer} \( C_G(x) \) of \( x \) in \( G \).
  It consists of all elements of \( G \) that commute with, i.e., centralize, \( x \).
  The intersection
  \[
    \bigcap_{x \in G} C_G(x) = \left\lbrace g \in G: gx = xg \text{ for all } x \in G \right\rbrace
  \]
  is the centre of \( G \).
\end{definition}

\begin{example}
  \label{example-stabilizer-of-coset}
  Let \( G \) act on \( G / H \) by left multiplication.
  Then \( \operatorname{Stab}(H) = H \) and the stabilizer of \( gH \) is \( gHg^{-1} \).
\end{example}

\begin{definition}
  \label{definition-stabilizer-of-subset}
  For a subset \( S \) of \( X \), we define the \emph{stabilizer} of \( S \) to be
  \[
    \operatorname{Stab}(S) = \left\lbrace g \in G: gS = S \right\rbrace.
  \]
\end{definition}
Like \ref{lemma-stabilizer-conjugate}, We also have
\[
    \operatorname{Stab}(gS) = g \cdot \operatorname{Stab}(S) \cdot g^{-1}.
\]

\begin{definition}
  \label{definition-normalizer}
  Let \( G \) act on \( G \) by conjugation, and let \( H \) be a subgroup of \( G \).
  The stabilizer of \( H \) is called the \emph{normalizer} \( N_G(H) \) of \( H \) in \( G \):
  \[
    N_G(H) = \left\lbrace g \in G: g H g^{-1} = H \right\rbrace.
  \]
  \( N_G(H) \) is the largest subgroup of \( G \) containing \( H \) as a normal subgroup.
\end{definition}


\begin{proposition}
  \label{proposition-isomorphic-G-set}
  If \( G \) acts transitively on \( X \), then for any \( x_0 \in X \), the map
  \[
    g \operatorname{Stab}(x_0) \mapsto gx_0: G / \operatorname{Stab}(x_0) \to X
  \]
  is an isomorphism of \( G \)-sets.
\end{proposition}

\begin{corollary}
  \label{corollary-orbit-cardinality}
  Let \( G \) act on \( X \), and let \( O = Gx_0 \) be the orbit containing \( x_0 \).
  Then the cardinality of \( O \) is
  \[
    \left\vert O \right\vert = (G: \operatorname{Stab}(x_0)).
  \]
\end{corollary}

\begin{proposition}
  \label{proposition-largest-subgroup-in-stabilizer}
  Let \( x_0 \in X \).
  If \( G \) acts transitively on \( X \), then
  \[
    \ker (G \to \operatorname{Sym}(X))
  \]
  is the largest normal subgroup contained in \( \operatorname{Stab}(x_0) \).
\end{proposition}

\begin{proof}
  It follows from the equation below and \ref{lemma-largest-normal-subgroup-in-subgroup}.
  \[
    \ker(G \to \operatorname{Sym}(X)) = \bigcap_{x \in X} \operatorname{Stab}(x) = \bigcap_{g \in G}\operatorname{Stab}(gx_0) = \bigcap g \cdot \operatorname{Stab}(x_0) \cdot g^{-1}.
  \]
\end{proof}


When \( X \) is finite, it is a disjoint union of a finite number of orbits
\[
  X = \bigcup_{i = 1}^m O_i
\]
Hence by \ref{corollary-orbit-cardinality}, we have the following results
\begin{proposition}
  \label{proposition-decomposition-of-set-cardinality}
  The number of elements in \( X \) is
  \[
    \left\vert X \right\vert = \sum_{i = 1}^m \left\vert O_i \right\vert = \sum_{i = 1}^m (G: \operatorname{Stab}(x_i)),\quad x_i \in O_i.
  \]
\end{proposition}

\begin{proposition}[Class Equation]
  \label{proposition-class-equation}
  \[
    \left\vert G \right\vert = \sum (G: C_G(x))
  \]
  where \( x \) runs over a set of representatives for the conjugacy classes, or
  \[
    \left\vert G \right\vert = \left\vert Z(G) \right\vert + \sum(G : C_G(y))
  \]
  where \( y \) runs over set of representatives for the conjugacy classes containing more than one element.
\end{proposition}

\begin{theorem}[Cauchy]
  \label{theorem-Cauchy-prime-order-element}
  If the prime \( p \) divides \( \left\vert G \right\vert \), then \( G \) contains an element of order \( p \).
\end{theorem}
\begin{proof}
  We use induction on \( \left\vert G \right\vert \).
  If for some \( y \) not in the centre of \( G \) and \( p \) does not divide \( (G: C_G(y)) \).
  Then \( p \) divides the order of \( C_G(y) \) and we apply induction.
  Thus we may suppose that \( p \) divides all of the terms \( (G: C_G(y)) \) in the class equation, and also divides \( Z(G) \).
  But \( Z(G) \) is commutative, and follows from the structure theorem of such groups that \( Z(G) \) will contain an element of order \( p \).
\end{proof}

\begin{corollary}
  \label{corollary-p-group-iff-condition}
  A finite group \( G \) is a \( p \)-group if and only if every element has order a power of \( p \).
\end{corollary}
\begin{proof}
  ``only if'' part follows from Lagrange's theorem \ref{theorem-Lagrange}.
  ``if'' part follows from Cauchy's theorem \ref{theorem-Cauchy-prime-order-element}: if not, suppose another \( p \neq p' \mid \left\vert G \right\vert \), then there exists an element \( a \in G \) with order \( p' \) contained in \( G \), a contradiction.
\end{proof}

\begin{corollary}
  \label{corollary-group-structure-with-order-2p}
  Every group of order \( 2p \), where \( p \) is odd prime, is cyclic or dihedral.
\end{corollary}
\begin{proof}
  From Cauchy's theorem \ref{theorem-Cauchy-prime-order-element}, we know that such a group \( G \) contains elements \( s \) and \( r \) of orders \( 2 \) and \( p \) respectively.
  Then \( H \) with index \( 2 \) is normal.
  Obviously, \( s \notin H \), and so \( G = H \cup Hs \):
  \[
    G = \left\lbrace 1, r, \cdots, r^{p - 1}, s, rs, \cdots, r^{p - 1}s \right\rbrace.
  \]
  As \( H \) is normal, \( srs^{-1} = r^i \) for some \( i \).
  Because \( s^2 = 1, r= s^2 r s^{-2} = s(srs^{-1})s^{-1} = r^{i^2} \), and so \( i^2 \equiv 1 \mod{p} \).
  Because \( \mathbb{Z} / p \mathbb{Z} \) is a field, its only elements with square \( 1 \) are \( \pm 1 \), and so \( i \equiv 1 \) or \( -1 \mod{p} \).
  In the first case, the group is commutative; in the second \( s r s^{-1} = r^{-1} \) and so it is the dihedral group.
\end{proof}


\begin{theorem}
  \label{theorem-finite-p-group-has-nontrivial-centre}
  Every nontrivial finite \( p \)-group has nontrivial centre.
\end{theorem}
\begin{proof}
  \ref{proposition-class-equation}
\end{proof}

\begin{corollary}
  \label{corollary-normal-subgroup-of-p-group}
  A group of order \( p^n \) has normal subgroups of order \( p^m \) for all \( m \leq n \).
\end{corollary}
\begin{proof}
  We use induction on \( n \).
  Let \( G \) be a group with order \( p^n \).
  By Cauchy'theorem \ref{theorem-Cauchy-prime-order-element}, \( Z(G) \) contains an element \( g \) of order \( p \), and so \( N = \left\langle g \right\rangle \) is a normal subgroup of \( G \) of order \( p \).
  It follows from the induction hypothesis to \( G / N \) and \ref{theorem-correspondence}.
\end{proof}

\begin{proposition}
  \label{proposition-structure-of-p^2-group}
  Every group of order \( p^2 \) is commutative, and hence is isomophic to \( C_p \times C_p \) or \( C_{p^2} \).
\end{proposition}
\begin{proof}
  \ref{lemma-structure-of-p^2-group}
\end{proof}

\begin{lemma}
  \label{lemma-structure-of-p^2-group}
  Suppose \( G \) contains a subgroup \( H \) in its centre(hence \( H \) is normal) such that \( G / H \) is cyclic.
  Then \( G \) is commutative.
\end{lemma}
\begin{proof}
  Let \( a \) be an element of \( G \) whose image in \( G / H \) generates \( it \).
  Then every element of \( G \) can be written \( g = a^i h \) with \( h \in H, i \in \mathbb{Z} \).
  Now
  \[
    a^i h \cdot a^{i'} h' = a^{i'} h' \cdot a^i h,
  \]
  by using the fact that \( H \subseteq Z(G) \).
\end{proof}

\subsection{Permutation Groups}
\label{subsection-permutation-groups}

\begin{definition}
  \label{definition-even-permutation}
  \label{definition-odd-permutation}
  \label{definition-permutation-signature}
  \label{definition-permutation-inversions}
  Consider \( \operatorname{Sym}(X) \), where \( X \) has \( n \) elements and consider a permutation
  \[
    \sigma = \begin{pmatrix}
      1 & 2 & 3 &\cdots &n\\
      \sigma(1) & \sigma(2) & \sigma(3) &\cdots & \sigma(n)
    \end{pmatrix}
  \]
  The ordered pairs \( (i, j) \) with \( i < j \) and \( \sigma(i) > \sigma(j) \) are called the \emph{inversions} of \( \sigma \), and \( \sigma \) is said to be \emph{even} or \emph{odd} according as the number its inversions is even or odd.
  The \emph{signature}, \( \operatorname{sign}(\sigma) \), of \( \sigma \) is \( + 1 \) or \( -1 \) according as \( \sigma \) is even or odd.
\end{definition}

\begin{proposition}
  \label{proposition-signature-as-homomorphism}
  \( \operatorname{sign}(\sigma) \operatorname{sign}(\tau) = \operatorname{sign}(\sigma \tau) \).
\end{proposition}
\begin{proof}
  For a permutation \( \sigma \), consider the products
  \[
    V = \prod_{1 \leq i < j \leq n} (j - i) = (2 - 1)(3 - 1) \cdots (n - 1)(3 - 2) \cdots (n - 2) \cdots (n - (n - 1))
  \]
  and
  \[
    \sigma V = \prod_{1 \leq i < j \leq n}(\sigma(j) - \sigma(i)).
  \]
  Both products run over the \( 2 \)-element subsets \( \left\lbrace i, j \right\rbrace \) of \( \left\lbrace 1, 2, \cdots, n \right\rbrace \) and the terms correponding to a subset are the same except that each inversion introduces a negative sgin.
  Therefore,
  \[
    \sigma V = \operatorname{sign}(\sigma)(V).
  \]

  Now let \( P \) be the additive group of maps \( \mathbb{Z}^n \to \mathbb{Z} \).
  For \( f \in P \) and \( \sigma \in S_n \), let \( \sigma f \) denote the element of \( P \) defined by
  \[
    (\sigma f)(z_1, \cdots, z_n) = f(z_{\sigma(1)}, \cdots, z_{\sigma(n)}).
  \]
  For \( z \in \mathbb{Z}^n \) and \( \sigma \in S_n \), let \( z^\sigma \) denote the element of \( \mathbb{Z}^n \) such that \( (z^\sigma)_i = z_{\sigma(i)} \).
  Then \( (z^\sigma)^\tau = z^{\sigma \tau} \).
  By definition, we have
  \[
    \sigma(\tau f) = (\sigma \tau)f.
  \]
  Let \( p \) be the element of \( P \) defined by
  \[
    p(z_1, \cdots, z_n) = \prod_{1 \leq i < j \leq n}(z_j - z_i).
  \]
  Then
  \[
    \sigma p = \operatorname{sign}(\sigma) p.
  \]
\end{proof}

\begin{definition}
  \label{definition-alternating-group}
  In \ref{proposition-signature-as-homomorphism}, we show that \( \operatorname{sign} \) is a homomorphism \( S_n \to \left\lbrace \pm 1 \right\rbrace \).
  When \( n \geq 2 \), it is surjective, and so its kernel is a normal subgroup of \( S_n \) or order \( n! / 2 \), called the \emph{alternating group} \( A_n \).
\end{definition}

\begin{definition}
  \label{definition-cycle}
  \label{definition-transposition}
  \label{definition-support-cycle}
  \label{definition-cycle-disjoint}
  A \emph{cycle} is a permutation of the following form
  \[
    i_1 \mapsto i_2 \mapsto i_3 \mapsto \cdots \mapsto i_r \mapsto i_1,\quad \text{ remaining } i \text{'s fixed.}
  \]
  The \( i_j \) are required to be distinct.
  We denote this cycle by \( (i_1 i_2 \cdots i_r) \), and call \( r \) its \emph{length}.
  A cycle of length \( 2 \) is a \emph{transposition}.
  A cycle of length \( 1 \) is the identity map.
  The \emph{support of the cycle} \( (i_1 \cdots i_r ) \) is the set \( \left\lbrace i_1, \cdots, i_r \right\rbrace \), and cycles are said to be \emph{disjoint} if their supports are disjoint.
\end{definition}
\begin{remark}
  \label{remark-disjoint-cycles-commute}
  Disjoint cycles commute.
  And if
  \[
    \sigma = (i_1 \cdots i_r)(j_1 \cdots j_s) \cdots (l_1 \cdots l_u)
  \]
  then
  \[
    \sigma^m = (i_1 \cdots i_r)^m(j_1 \cdots j_s)^m \cdots (l_1 \cdots l_u)^m
  \]
  and it follows that \( \sigma \) has order \( \operatorname{lcm}(r,s,\cdots, u) \).
\end{remark}

\begin{proposition}
  \label{proposition-permutation-as-disjoint-cycles}
  Every permutation can be written as a product of disjoint cycles.
\end{proposition}
\begin{example}
  \label{example-permutation-as-disjoint-cycles}
  \[
    \begin{pmatrix}
    1 & 2 & 3 & 4 & 5 & 6 & 7 & 8\\
    5 & 7 & 4 & 2 & 1 & 3 & 6 & 8
    \end{pmatrix} = (15)(27634)(8).
  \]
\end{example}

\begin{corollary}
  \label{proposition-permutation-as-transpositions}
  Each permutation \( \sigma \) can be written as a product of transpositions;
  the number of transpositions in such a product is even or odd according as \( \sigma \) is even or odd.
  In particular, the signature of a cycle of length \( r \) is \( (-1)^{r - 1} \), that is, an \( r \)-cycle is even or odd according as \( r \) is odd or even.
\end{corollary}
\begin{proof}
  Noting that \( (i_1 i_2 \cdots i_r) = (i_1 i_2) \cdots (i_{r - 2}i_{r - 1})(i_{r - 1} i_r) \).
\end{proof}

\begin{corollary}
  \label{corollary-alternating-group-generated-by-3-cycles}
  The alternating group \( A_n \) is generated by cycles of length three.
\end{corollary}
\begin{proof}
  \[
    (ij)(kl) = \begin{cases}
      (ij) (jl) = (ijl) & j = k,\\
      (ij)(jk)(jk)(kl) = (ijk)(jkl) & i,j,k,l \text{ distinct }\\
      1 & (ij) = (kl)
    \end{cases}
  \]
\end{proof}


In \( S_n \), the conjugate of a cycle is given by
\[
  g(i_1 \cdots i_k)g^{-1} = (g(i_1) \cdots g(i_k)).
\]
We shall determine the conjugacy classes in \( S_n \).

\begin{definition}
  \label{definition-number-partition}
  By a \emph{partition} of \( n \), we mean a sequence of integers \( n_1, \cdots, n_k \) such that
  \[
    1 \leq n_1 \leq n_2 \leq \cdots n_k \leq n \text{ and } n_1 + n_2 + \cdots + n_k = n.
  \]
\end{definition}

\begin{proposition}
  \label{proposition-conjugate-iff-condition}
  Two elements \( \sigma \) and \( \tau \) of \( S_n \) are conjugate if and only if they define the same partitions of \( n \).
\end{proposition}
\begin{proof}
  \( \impliedby \): Since \( \sigma \) and \( \tau \) define the same partitions of \( n \), their decompositions into products of disjoint cycles have the same type:
  \begin{align*}
    \sigma = (i_1 \cdots i_r)(j_1 \cdots j_s)\cdots(l_1 \cdots l_u),\\
    \tau = (i'_1 \cdots i'_r)(j'_1 \cdots j'_s)\cdots(l'_1 \cdots l'_u).
  \end{align*}
  If we define \( g \) to be
  \[
    \begin{pmatrix}
      i_1 &\cdots &i_r &j_1 &\cdots &j_s &\cdots &l_1 &\cdots &l_u\\
      i'_1 &\cdots &i'_r &j'_1 &\cdots &j'_s &\cdots &l'_1 &\cdots &l'_u
    \end{pmatrix}
  \]
\end{proof}
\begin{remark}
  \label{remark-number-in-conjugacy-class}
  For \( 1 < k \leq n \), there are \( \frac{n(n - 1) \cdots (n - k + 1)}{k} \) distinct \( k \)-cycles in \( S_n \).
  The \( 1 / k \) is needed so that we don't count
  \[
    (i_1 i_2 \cdots i_k) = (i_k i_1 \cdots i_{k - 1}) = \cdots
  \]
  \( k \) times.
  Similarly, it is possible to compute the number of elements in any conjugacy class in \( S_n \), but a little care is needed when the partition of \( n \) has several terms equal.
  For example, the number of permutation in \( S_4 \) of type \( (ab)(cd) \) is
  \[
    \frac{1}{2}\left(\frac{4 \cdot 3}{2} \cdot \frac{2 \cdot 1}{2}\right) = 3.
  \]
  The \( \frac{1}{2} \) is needed so that we don't count \( (ab)(cd) = (cd)(ab) \) twice.
  For \( S_4 \) we have the follow table:
  \begin{table}[H]
    \centering
    \begin{tabular}{cccc}
      partition & element & No. in Conj. Class & Parity\\
      \( 1 + 1 + 1 + 1 \) & 1 & 1 & even\\
      \( 1 + 1 + 2 \) & \( (ab) \) & 6 & odd\\
      \( 1 + 3 \) & \( (abc) \) & 8 & even\\
      \( 2 + 2 \) & \( (ab)(cd) \) & 3 & even\\
      \( 4 \) & \( (abcd) \) & 6 & odd
    \end{tabular}
  \end{table}
  Note that \( A_4 \) contains exactly \( 3 \) elements of order \( 2 \), namely those of type \( 2 + 2 \), and that with \( 1 \) they form a subgroup \( V \).
  This group is a union of conjugacy classes, and is therefore a normal subgroup of \( S_4 \).
\end{remark}


\begin{theorem}[Galois]
  \label{theorem-Galois-alternating-group}
  The group \( A_n \) is simple if \( n \geq 5 \).
\end{theorem}
\begin{proof}
  \ref{lemma-Galois-alternating-group-1} and \ref{lemma-automorphisms-of-cyclic-group-2}.
\end{proof}

\begin{remark}
  \label{remark-alternating-group-with-order-2-and-3}
  For \( n = 2 \), \( A_n \) is trivial, and for \( n = 3 \), \( A_n \) is cyclic of order \( 3 \), and hence simple.
\end{remark}

\begin{lemma}
  \label{lemma-Galois-alternating-group-1}
  Let \( N \) be a normal subgroup of \( A_n(n \geq 5) \);
  if \( N \) contains a cycle of length three, then it contains all cycles of length three, and so equal \( A_n \).
\end{lemma}
\begin{proof}
  Let \( \gamma \) be the cycle of length three in \( N \), and let \( \sigma \) be a second cycle of length three in \( A_n \).
  We know that \( \sigma = g \gamma g^{-1} \) for some \( g \in S_n \).
  \begin{itemize}
    \item If \( g \in A_n \), then this shows that \( \sigma \) is also in \( N \).
    \item If not, because \( n \geq 5 \), there exists a transposition \( t \in S_n \) disjoint from \( \sigma \).
      Then \( tg \in A_n \), and
      \[
        \sigma = t \sigma t^{-1} = tg \gamma g^{-1} t^{-1},
      \]
      and so again \( \sigma \in N \).
  \end{itemize}
\end{proof}

\begin{lemma}
  \label{lemma-Galois-alternating-group-2}
  Every normal subgroup \( N \) of \( A_n \), \( n \geq 5 \), \( N \neq 1 \), contains a cycle of length \( 3 \).
\end{lemma}
\begin{proof}
  Let \( \sigma \in N \), \( \sigma \neq 1 \).
  If \( \sigma \) is not a \( 3 \)-cycle, then \( \sigma' \neq 1 \), which fixes more elements of \( \left\lbrace 1, 2, \cdots, n \right\rbrace \) than does \( \sigma \).
  If \( \sigma' \) is not, we can apply the same construction.
  After a finite number of steps, we arrive at a \( 3 \)-cycle.

  Suppose \( \sigma \) is not a \( 3 \)-cycle.
  When we express it as a product of disjoint cycles, either it contains a cycle of length \( \geq 3 \) or else it is a product of transpositions.
  \begin{enumerate}
    \item \( \sigma = (i_1 i_2 i_3 \cdots) \cdots \).
      \( \sigma \) moves two numbers, say \( i_4, i_5 \) other than \( i_1, i_2, i_3 \) since \( \sigma \neq (i_1 i_2 i_3), (i_1 \cdots i_4) \).
      Let \( \gamma = (i_3 i_4 i_5) \), then \( \sigma_1 := \gamma \sigma \gamma^{-1} = (i_1 i_2 i_4 \cdots) \cdots \in N \), and is distinct from \( \sigma \).
      Thus \( \sigma' := \sigma_1 \sigma^{-1} \neq 1 \), but \( \sigma' = \gamma \sigma \gamma^{-1} \sigma^{-1} \) fixes \( i_2 \) and all elements other than \( i_1, \cdots, i_5 \) fixed by \( \sigma \).
      Therefore, it fixes more elements than \( \sigma \).
    \item \( \sigma = (i_1 i_2)(i_3 i_4) \cdots \).
      form \( \gamma, \sigma_1, \sigma' \) as the first case with \( i_4 \) as in (2) and \( i_5 \) any element distinct from \( i_1, i_2, i_3, i_4 \).
      Then \( \sigma_1 = (i_1 i_2)(i_4 i_5) \cdots \) is distinct from \( \sigma \) because it acts differently on \( i_4 \).
      Thus \( \sigma' = \sigma_1 \sigma^{-1} \neq 1 \), but \( \sigma' \) fixes \( i_1 \) and \( i_2 \), and all elements \( \neq i_1, \cdots, i_5 \) not fixed by \( \sigma \).
      Therefore it fixes at least one more element than \( \sigma \).
  \end{enumerate}
\end{proof}

\begin{corollary}
  \label{corollary-normal-subgroup-of-symmetric-group}
  For \( n \geq 5 \), the only normal subgroups of \( S_n \) are \( 1 \), \( A_n \), and \( S_n \).
\end{corollary}
\begin{proof}
  If \( N \) is normal in \( S_n \), then either \( N \cap A_n = A_n \) or \( N \cap A_n = \left\lbrace 1 \right\rbrace \).
  In the second case, the map \( x \mapsto x A_n: N \to S_n / A_n \) is injective, but it can't have order \( 2 \) because no conjugacy class in \( S_n \) consists of a single element.
\end{proof}

\subsection{The Todd-Coxeter algorithm}
\label{subsection-the-Todd-Coxeter-algorithm}

Let \( G \) be a group described by a finite presentation, and let \( H \) be a subgroup described by a generated set.
Then the \emph{Todd-Coxeter algorithm} is a strategy for writing down the set of left cosets of \( H \) in \( G \) together with the action of \( G \) on the set.

Let \( G = \left\langle a, b, c \mid a^3, b^2, c^2, cba \right\rangle \) and let \( H \) be the subgroup generated by \( c \).
The operation of \( G \) on the set of cosets is described by the action of generators which must satisfy the following rules
\begin{enumerate}
  \item Each generator acts as a permutation.
  \item The relations act trivially.
  \item The generators of \( H \) fix the coset \( 1H \).
  \item The operation on the cosets is transitive.
\end{enumerate}

\subsection{Primitive actions}
\label{subsection-primitive-actions}

\begin{definition}
  \label{definition-stabilized-partition}
  Let \( G \) be a group acting on a set \( X \), and let \( \pi \) be a partition of \( X \).
  We say that \( \pi \) is \emph{stabilized} by \( G \) if
  \[
    A \in \pi \implies gA \in \pi.
  \]
\end{definition}

\begin{example}
  \label{example-stabilized-action-on-1234}
  \begin{enumerate}
    \item The subgroup \( G = \left\langle (1234) \right\rangle \) of \( S_4 \) stabilizes the partition \( \left\lbrace \left\lbrace 1, 3 \right\rbrace, \left\lbrace 2, 4 \right\rbrace \right\rbrace \) of \( \left\lbrace 1, 2, 3, 4 \right\rbrace \).
    \item Identify \( X = \left\lbrace 1, 2, 3, 4 \right\rbrace \) with the set of vertices of the square on which \( D_4 \) acts in the usual way, namely, with \( r = (1234) \), \( s = (24) \).
      Then \( D_4 \) stabilizes the partition \( \left\lbrace \left\lbrace 1, 3 \right\rbrace, \left\lbrace 2, 4 \right\rbrace \right\rbrace \)(opposite vertices stay opposite).
  \end{enumerate}
\end{example}

\begin{definition}
  \label{definition-primitive-action}
  \label{definition-imprimitive-action}
  The group \( G \) always stabilizes the trivial partitions of \( X \), namely, the set of all one-element subsets of \( X \), and \( \left\lbrace X \right\rbrace \).
  When it stabilizes only those partitions, we say that the action is \emph{primitive}; otherwise it is \emph{imprimitive}.
  A subgroup of \( \operatorname{Sym}(X) \) is said to be \emph{primitive} if it acts primitively on \( X \).
\end{definition}

\begin{example}
  \label{example-symmetric-group-primitive}
  \( S_n \) itself is primitive.
\end{example}

\begin{example}
  \label{example-doubly-transitive-primitive}
  A doubly transitive action is primitive: if it stabilized
  \[
    \left\lbrace \left\lbrace x, x' \right\rbrace, \left\lbrace y, \cdots, \right\rbrace, \cdots \right\rbrace,
  \]
  then there would be no element sending \( (x, x') \) to \( (x, y) \).
\end{example}


\begin{proposition}
  \label{proposition-imprimitive-iff-condition}
  Let \( G \) be a finite group acting transitively on a set \( X \) with at least two elements.
  The group \( G \) acts imprimitively \( \iff \) there is a proper subset \( A \) of \( X \) with at least \( 2 \) elements such that
  \begin{equation}
    \text{for each } g \in G, \text{ either } gA = A \text{ or } gA \cap A = \emptyset. \label{equation-block}
  \end{equation}
\end{proposition}
\begin{proof}
  \( \impliedby \): From such an \( A \), we can form a partition \( \left\lbrace A, g_1 A, g_2 A, \cdots \right\rbrace \) of \( X \), which is stabilized by \( G \)(Recall that we assume \( G  \) acts transitively on \( X \)).
\end{proof}

\begin{definition}
  \label{definition-block}
  Let \( G \) be a finite group acting transitively on a set \( X \) with at least two elements.
  A subset \( A \) of \( X \) satisfying \ref{equation-block} is called \emph{block}.
\end{definition}

\begin{lemma}
  \label{lemma-primitive-if-condition}
  Let \( G \) be a finite group acting transitively on a set \( X \) with at least two elements.
  Let \( A \) be a block in \( X \) with \( \left\vert A \right\vert \geq 2 \) and \( A \neq X \).
  For any \( x \in A \),
  \[
    \operatorname{Stab}(x) \subsetneq \operatorname{Stab}(A) \subsetneq G.
  \]
\end{lemma}
\begin{proof}
  \( \operatorname{Stab}(A) \supseteq \operatorname{Stab}(x) \) because
  \[
    gx = x \implies gA \cap A \neq \emptyset \implies gA = A.
  \]
  Let \( y \in A, y \neq x \).
  Because \( G \) acts transitively on \( X \), there is a \( g \in G \) such that \( gx = y \).
  Then \( g \in \operatorname{Stab}(A) \), but \( g \notin \operatorname{Stab}(x) \).
  Let \( y \notin A \).
  There is a \( g \in G \) such that \( gx = y \), and then \( g \notin \operatorname{Stab}(x) \).
\end{proof}

\begin{theorem}
  \label{theorem-primitive-if-condition}
  Let \( G \) be a finite group acting transitively on a set \( X \) with at least two elements.
  The group \( G \) acts primitively on \( X \iff \) for one \( x \in X, \operatorname{Stab}(x) \) is a maximal subgroup(hence any) of \( G \).
\end{theorem}
\begin{proof}
  \( \impliedby \) follows from \ref{lemma-primitive-if-condition}.
  \( \implies \): suppose that there exists an \( x \) in \( X \) and a subgroup \( H \) such that
  \[
    \operatorname{Stab}(x) \subsetneq H \subsetneq G.
  \]
    Then we claim that \( A = Hx \) is a block \( \neq X \) with at least two elements.
    Because \( H \neq \operatorname{Stab}(x), Hx \neq \left\lbrace x \right\rbrace \), and so \( \left\lbrace x \right\rbrace \subsetneq A \subsetneq X \).
    If \( g \in H \), then \( gA = A \).
    If \( g \notin H \), then \( gA \) is disjoint from \( A \): for suppose \( ghx = h' x \) for some \( h' \in H \);
    then \( h'^{-1}gh \in \operatorname{Stab}(x) \subseteq H \), say \( h'^{-1}gh = h'' \), and \( g = h'h''h^{-1} \in H \).
\end{proof}

\subsection{Sylow Theorem}
\label{subsection-sylow-theorem}

In this subsection, all group are finite.
\begin{definition}
  \label{definition-Sylow-p-subgroup}
  Let \( G \) be a group and let \( p \) be a prime dividing \( (G: 1) \).
  A subgroup of \( G \) is called a \emph{Sylow} \emph{\( p \)-subgroup of} \( G \) if its order is the highest power of \( p \) dividing \( (G: 1) \).
\end{definition}


In the proofs, we frequently use that if \( O \) is an orbit for a group \( H \) acting on a set \( X \), and \( x_0 \in O \), then the map \( H \to X, h \mapsto h x_0 \) induces a bijection
\[
  H / \operatorname{Stab}(x_0) \to O;
\]
Therefore
\[
  (H: \operatorname{Stab}(x_0)) = \left\vert O \right\vert.
\]
In particular, when \( H \) is a \( p \)-group, \( \left\vert O \right\vert \) is a power of \( p \), and so either \( O \) consists of a single element, or \( \left\vert O \right\vert \) is divisible by \( p \).

\begin{theorem}[Sylow I]
  \label{theorem-sylow-I}
  Let \( G \) be a finite group, and let \( p \) be prime, then \( G \) has a subgroup of order \( p^r \).
\end{theorem}
\begin{proof}
  It suffices to prove this with \( p^r \) the highest power of \( p \) dividing \( (G : 1) \), and so from now on we assume that \( (G: 1) = p^r m \) with \( p \nmid m \).
  Let
  \[
    X = \left\lbrace \text{subsets of } G \text{ with } p^r \text{ elements} \right\rbrace,
  \]
  with the action of \( G \) defined by
  \[
    G \times X \to X,\quad (g, A) \mapsto gA.
  \]
  Let \( A \in X \), and let
  \[
    H = \operatorname{Stab}(A) \coloneq \left\lbrace g \in G: gA = A \right\rbrace.
  \]
  For any \( a_0 \in A, h \mapsto h a_0: H \to A \) is injective, since \( A \subseteq G \).
  And so \( (H : 1) \leq \left\vert A \right\vert = p^r \).
  In the equation
  \[
    (G : 1) = (G : H)(H : 1)
  \]
  we know that \( (G : 1) = p^r m, (H : 1) \leq p^r \) and that \( (G : H) \) is the number of elements in the orbits of \( A \).
  Observe that: if we can find an \( A \) such that \( p \) doesn't divide the number of elements in its orbit, then we can conclude that \( H = \operatorname{Stab}A \) has order \( p^r \).
  The number of elements in \( X \) is
  \[
    \left\vert X \right\vert = \binom{p^rm}{p^r} = \frac{(p^rm)(p^r m - 1) \cdots (p^rm- i) \cdots (p^rm - p^r + 1)}{p^r(p^r - 1) \cdots (p^r - i)\cdots (p^r - p^r + 1)}.
  \]
  Note that, because \( i < p^r \), the power of \( p \) dividing \( p^r m - i \) is the power of \( p \) dividing \( i \).
  The same is true for \( p^r - i \).
  Therefore the correponding terms on top and bottom are divisible by the same powers of \( p \), and so \( p \) does not divide \( \left\vert X \right\vert \).
  Because the orbits form a partition of \( X \),
  \[
    \left\vert X \right\vert = \sum \left\vert O_i \right\vert, \quad O_i\text{ the distinct orbits }.
  \]
  and so at least one of the \( \left\vert O_i \right\vert \) is not divisible by \( p \).
\end{proof}

\begin{lemma}
  \label{lemma-Sylow-II-1}
  Let \( H \) be a \( p \)-group acting on a finite set \( X \), and let \( X^H \) be the set of points fixed by \( H \);
  then
  \[
    \left\vert X \right\vert \equiv \left\vert X^H \right\vert \pmod{p}.
  \]
\end{lemma}
\begin{proof}
  \ref{proposition-decomposition-of-set-cardinality}.
\end{proof}

\begin{lemma}
  \label{lemma-Sylow-II-2}
  Let \( P \) be a Sylow \( p \)-subgroup of \( G \), and let \( H \) be a \( p \)-subgroup.
  If \( H \) normalizes \( P \), i.e., if \( H \subseteq N_G(P) \), then \( H \subseteq P \).
  In particular, no Sylow \( p \)-subgroup of \( G \) other than \( P \) normalizes \( P \).
\end{lemma}
\begin{proof}
  Because \( H \) and \( P \) are subgroups of \( N_G(P) \) with \( P \) normal in \( N_G(P) \), \( HP \) is a subgroup, and \( H / H \cap P \simeq HP / P \)
  Therefore \( (HP : P) \) is a power of \( p \), but
  \[
    (HP : 1) = (HP : P)(P : 1),
  \]
  and \( (P : 1) \) is the largest power of \( p \) dividing \( (G : 1) \), hence also the largest power of \( p \) dividing \( (HP : 1) \).
  Hence \( (HP : P) = 1 \), and \( H \subseteq P \).
\end{proof}

\begin{theorem}[Sylow II]
  \label{theorem-Sylow-II}
  Let \( G \) be a finite group, and let \( \left\vert G \right\vert = p^r m \) with \( m \) not divisible by \( p \).
  \begin{enumerate}
    \item Any two Sylow \( p \)-subgroups are conjugate.
    \item Let \( s_p \) be the number of Sylow \( p \)-subgroups in \( G \);
      then \( s_p \equiv 1 \mod{p} \) and \( s_p \mid m \).
    \item Every \( p \)-subgroup of \( G \) is contained in a Sylow \( p \)-subgroup.
  \end{enumerate}
\end{theorem}

\begin{proof}
  \begin{enumerate}
    \item Let \( X \) be the set of Sylow \( p \)-subgroups in \( G \), and let \( G \) ac on \( X \) by conjugation,
      \[
        (g, P) \mapsto g P g^{-1}: \quad G \times X \to X.
      \]
      Let \( O \) be one of the \( G \)-orbits: we have to show \( O \) is all of \( X \).

      Let \( P \in O \), and let \( P \) act on \( O \) through the action of \( G \).
      This single \( G \)-orbit may break up into serveral \( P \)-orbits, one of which will be \( \left\lbrace P \right\rbrace \).
      In fact this is the only one-point orbit because
      \[
        \left\lbrace Q \right\rbrace \text{ is a } P \text{-orbit} \iff P \text{ normalizes } Q.
      \]
      We know that happens only for \( Q = P \) by \ref{lemma-Sylow-II-2}.
      Hence the number of elements in every \( P \)-orbit other than \( \left\lbrace P \right\rbrace \) is divisible by \( p \), and we have that \( \left\vert O \right\vert \equiv 1 \mod{p} \).

      Suppose there exists a \( P \notin O \).
      We again let \( P \) act on \( O \), but this time the argument shows that there are no one-point orbit, and so the number of elements in every \( P \)-orbit is divisible by \( p \)(the orbit equation).
      This implies that \( \# O \) is divisible by \( p \), which is a contradiction.
    \item Let \( P \) be a Sylow \( p \)-subgroup of \( G \).
      We have shown that \( s_p \equiv 1 \pmod{p} \).
      Then
      \[
        s_p = (G : N_G(P)) = \frac{(G : 1)}{(N_G(P) : 1)} = \frac{(G : 1)}{(N_G(P): P) \cdot (P : 1)} = \frac{m}{(N_G(P): P)}.
      \]
    \item Let \( H \) be a \( p \)-subgroup of \( G \), and let \( H \) act on the set \( X \) of Sylow \( p \)-subgroups by conjugation.
      Because \( \left\vert X \right\vert = s_p \) is not divisible by \( p \), \( X^H \) must be nonemepty by \ref{lemma-Sylow-II-1}.
      But then \( H \) normalizes \( P \) and the preceding lemma implies that \( H \subseteq P \).
  \end{enumerate}
\end{proof}

\begin{corollary}
  \label{corollary-Sylow-p-subgroup-normal}
  A Sylow \( p \)-subgroup is normal \( \iff \) it is only Sylow \( p \)-subgroup.
\end{corollary}

\begin{corollary}
  \label{corollary-direct-product-of-Sylow-p-subgroups}
  Suppose that a group \( G \) has only one Sylow \( p \)-subgroup for each prime \( p \) dividing its order.
  Then \( G \) is a direct product of its Sylow \( p \)-subgroups.
\end{corollary}
\begin{proof}
  Let \( P_1, \cdots, P_k \) be Sylow subgroups of \( G \), and let \( \left\vert P_i \right\vert = p^{r_i}_i \), where the \( p_i \) are distinct primes.
  We shall prove by induction on \( k \) that it has order \( p^{r_1}_1 \cdots p^{r_k}_k \).
  We may suppose that \( k \geq 2 \) and \( P_1 \cdots P_{k - 1} \) has order \( p^{r_1}_1\cdots p^{r_{k - 1}}_{k - 1} \).
  Then \( P_1 \cdots P_{k - 1} \cap P_k = 1 \) then \( (P_1 \cdots P_{k - 1})P_k \) is the direct product of \( P_1 \cdots P_{k - 1} \) and \( P_k \), and so has order \( p^{r_1}_1 \cdots p^{r_k}_k \).
\end{proof}

\end{document}
